\chapter{Outlook}\label{chapter:outlook}

This study had the following limitations and future work can be done in the following areas:

\begin{itemize}
    \item As sub-pixel algorithms were not implemented in this research, a comparison survey can be done between different algorithms. Sub-Pixel interpolation schemes such as 3 point guassian interpolation and high order interpolation scheme were highly recommended by Schreier et al. \cite{schreier} and Knauss et al. \cite{knauss} as they provided better convergence than the other interpolation schemes \cite{pan_review}.
    \item The efficacy of \gls{mig} in accuracy of measured displacement on extreme cases of fine speckle size and high speckle density is worth for further study.
    \item As \gls{ncc} cannot measure rotation, feature tracking approaches that measure rotation can be tested  in real applications.
    \item The current study was done only for small distances. Further research can be done, in implementing schemes that allow measuring longer distances. 
    \item Velocity measurement analysis was not performed for this research. Furthermore, the \gls{lsp} images were only captured at low speeds of \SI{1}{\milli\meter/\second}. The opportunity to perform velocimetry using hexapods capable of higher velocities remains an open possibility.  
\end{itemize}
    
  
% \section*{Limitations and Future Scope}  

% \begin{itemize}
%     \item \gls{mig} fails to take speckle size into account. Other than that, the range determined for \gls{mig} is subject to change due to environmental factors, for example temperature of camera. Hence, an accurate assessment of \gls{mig} range is not performed. The assumption here is, that for most reliable measurements, attempt should be made for highest possible \gls{mig} value. Other kinds of image quality parameters remain to be tested as well. 
%     \item The position error analysis is only done for the planar XY movements. But, in a real sceanrio using a robot, there will also be vibrations in Z-Axis. Due to time constraints, no experiments are performed to test this.
%     \item The calibration process is in itself affected by pixel-locking and vibrations. These inaccuracies make their way into the final position calculation. Therefore, a cumulative effect of error is happening, as the actual testing is also marred by these issues. Better signal processing methods need to be researched, so that errors from the calibration error can be minimized. 
%     \item To obtain sub-pixel accuracy it involves two steps: accurate initial guess of displacement and then sub-pixel interpolation methods to improve accuracy.
%     \item The disadvantage of NCC is outlined in the paper by Pan et. al. \cite{pan}.
%     \item Due to nature of \gls{ncc} algorithm, it is not possible to test \gls{lsp} for rotation. Due to time constraints, other approaches which take rotation of \gls{roi} template into account are not tested.
%     \item The position error analysis is only done on small distances. For longer distances, template resetting becomes necessary. It involves resetting the \gls{roi} used for template matching. This approach will allow looking into a possible issue of error accumulation during longer translations.
%     \item Velocity verification is also not performed due to time constraints. Other than that, the project findings are collected using a hexapod, which has a maximum velocity of \SI{1}{\milli\meter/\second}. The opportunity to test relative position and velocity using \gls{lsp} captured with higher velocity hexapod still remains a possibility.
% \end{itemize}
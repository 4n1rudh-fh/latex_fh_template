\chapter{Experimental Design}

As mentioned in Table \ref{table:technical_specifications_hexapod}, the hexapod is able to move to a certain user-given coordinate with a precision of \SI{\pm0.5}{\micro\meter} accuracy. Thus, the goals of testing were:
\begin{itemize}
    \item to verify whether the camera, laser module along with \gls{ncc} is able to determine this position,
    \item and to what accuracy in comparison to hexapod position.     
\end{itemize}

\vspace{5mm}

\noindent In order to achieve the testing goal, the procedure can be divided into three parts.

\begin{itemize}
    \item \textbf{Setting up code logic} performing \gls{ncc}.
    \item \textbf{Calibration matrix} to convert \gls{ncc} algorithm results to actual distance.
    \item \textbf{Dataset collection} for the two setups outlined in Fig. \ref{fig:actual_old_setup.png} and Fig. \ref{fig:actual_corrected_setup.png}.
\end{itemize}


\section{Setting up Code Logic}
    \subsection*{Centre Frame}
        The camera image sensor has rectangular dimensions of 728 x 544 as given in Table \ref{table:camera_specs}. For the way, the camera is mounted for the experiments, it meant that the width of image was more than it's height. The example image is given in Fig. \ref{fig:insert_figure.png}. It became necessary going forward keeping \emph{template re-referencing} (See Appendix.) in mind for future scope, that a square \gls{roi} at the center of this image frame with dimensions of 128 pixels x 128 pixels would be used as a template for \gls{ncc} going forward. This size for \gls{roi} was chosen as:
        \begin{itemize}
            \item it is not too small to lose unique information inside it and the template is still tracked to reasonable accuracy,
            \item but also not too large to slow the process of template matching.
        \end{itemize}            
    \subsection*{Realtive Position Measurement}
        \gls{ncc} with OpenCV, when tracks the template inside an image, it provides the location of template's top-left corner. The coordinate system for images in OpenCV by default is shown in the Fig. \ref{fig:insert_fig.png}. \textbf{Caution:} Research on how OpenCV converts the top-left location to center in cmap. In order to measure relative movement, the relative pixel shift of template was based on the center of this image.

\vspace{5mm}

\noindent Going forward with this idea, it was important to decide on experimental factors that were going to remain constant during the testing process. How do parameters such as laser diameter, camera parameters such as gain and exposure time and image quality parameters such as \gls{mig} have on the positional measurement using a camera, \gls{ncc} and laser. 

\vspace{5mm}

\noindent The \emph{f}/1.85 aperture was used for the camera. As shown in Fig. \ref{fig:corrected_setup.png} the working surface was approximately \SI{150}{\milli\meter} below the image sensor location inside the camera. The laser beam diameter for this height was approximately \SI{7}{\milli\meter} at the working surface. The experimental design was divided into the following:
\begin{itemize}
    \item Setup 1
    \item Calibration
    \item Testing
    \item Setup 2
    \item Calibration
    \item Testing
\end{itemize}

\section{Calibration}
In the \gls{ncc} algorithm employed using OpenCV, one gets location of the template's top-left corner, that is being detected in the given frame. The code employed for the project converts this absolute location of the top-left corner into relative movement as seen from the center of the frame. Explain \texbf{relative pixel movement} and why we chose the \textbf{center of the frame}.


\section{Testing}

\section*{To-Do}
\begin{itemize}
    \item \sout{Say here that the aperture size was this and this.}
    \item \sout{Do I say something about the corrected as well as old setup here?}
    \item \sout{What reason should I provide for the old setup?}
    \item Why did I only do x, y movement for the hexapod? Why was z movement not performed?
    \item Why did we only move 0.3 mm?
    \item Because we are moving 0.3 mm, the repeatability error of hexapod with value \SI{\pm0.5}{\micro\meter} can be ignored? How much is the percentage of error? 0.1667\%
    \item Put the quadrant diagram for performed movements here.
    \item Why rotations were not done? My answer because of NCC.
    \item Put research questions for the next chapter.
    \item \sout{Talk about laser diameter used for our experiments.}
    \item Have you taken into account every parameter you wanna talk about?
    \item Explain transformation matrix. 
\end{itemize}
\chapter{Experimental Design}

As mentioned in Table \ref{table:technical_specifications_hexapod}, the hexapod is able to move to a certain user-given coordinate with a precision of \SI{\pm0.5}{\micro\meter} accuracy. Thus, the goals of testing are:
\begin{itemize}
    \item to verify whether the camera, laser module along with \gls{ncc} is able to determine this position,
    \item and to what accuracy in comparison to hexapod position.     
\end{itemize}

\vspace{5mm}

\noindent In order to achieve the testing goal, the procedure can be divided into three parts.

\begin{itemize}
    \item \textbf{Setting up code logic} performing \gls{ncc}.
    \item \textbf{Calibration matrix} to convert \gls{ncc} algorithm results to actual distance.
    \item \textbf{Dataset collection} for the two setups outlined in Fig. \ref{fig:actual_old_setup.png} and Fig. \ref{fig:actual_corrected_setup.png}.
    \item \textbf{Constant parameters} during testing
\end{itemize}


\section{Setting up Code Logic}\label{section:code_logic}
    \subsection*{Centre Frame}
        The camera image sensor has rectangular dimensions of 728 x 544 as given in Table \ref{table:camera_specs}. The way the camera is mounted for the experiments, it meant that the width of image was more than it's height. It also became necessary going forward, that a square \gls{roi} at the center of this image frame with dimensions of 128 pixels x 128 pixels would be used as a template for \gls{ncc} going forward. The reason is, for position verification of the hexapod, the data collection procedure made it necessary to use the \gls{roi} at the center of the image frame (See Section \ref{section:data_collection}). This is denoted by the square box outlined in black in Fig. \ref{fig:frame_0_rect.png}. The size for \gls{roi} was chosen for the following reasons:
        \begin{itemize}
            \item it is not too small to lose unique information inside it and the template is still tracked to reasonable accuracy,
            \item but also not too large to slow the process of template matching.
        \end{itemize}
        
    \subsection*{Relative Position Measurement}
        In OpenCV, as shown in Fig. \ref{fig:frame_0_rect.png}, for a coordinate (X, Y), X increases as we are going towards the right and Y increases as we traverse through the pixels downwards. For the project, while using \gls{ncc} with OpenCV two functions were employed namely, \texttt{matchTemplate()} and \texttt{minMaxLoc()}. The former function matches the template with the underlying image and saves it's results in another matrix. The latter function allows us to use the result from first function in order to find the location of correlation peak. In doing so, it gives the absolute location of top-left corner of the tracked template according to the OpenCV coordinate axes. Therefore, in order to measure relative movement from the center of the image, changes were made to reflect pixel shift with respect to the center of the \gls{roi}. For e.g., if the hexapod moves from (\SI{0}{\milli\meter}, \SI{0}{\milli\meter}) to (\SI{3}{\milli\meter}, \SI{3}{\milli\meter}) in the first quadrant, the template is tracked in the the fourth quadrant. This is shown by the arrow in Fig. \ref{fig:frame_60_rect.png}. 

        \begin{figure}[h]
            \centering
            \includegraphics[width=0.68\textwidth]{images/e_exp_design/frame_0_rect.png}
            \caption{\gls{lsp} showing OpenCV coordinate axes.}
            \label{fig:frame_0_rect.png}
        \end{figure}
        
        \begin{figure}[h]
            \centering
            \includegraphics[width=0.7\textwidth]{images/e_exp_design/frame_60_rect.png}
            \caption{\gls{lsp} showing relative movement with respect to center of frame.}
            \label{fig:frame_60_rect.png}
        \end{figure}

\vspace{5mm}

\section{Calibration}\label{section:calibration}
Discuss calibration procedure here from the Prof. Charrett paper. 


\section{Data Collection}\label{section:data_collection}
Put the quadrant design here. Denoting the movement name with respect to quadrants. The calibration direction. How each setup has different calibration and experiment numbers. Explain also why did we only move 0.3 mm.

\section{Constant Parameters}\label{section:constant_parameters}

\noindent Going forward with this idea, it was important to decide on experimental factors that were going to remain constant during the testing process. How do parameters such as laser diameter, camera parameters such as gain and exposure time and image quality parameters such as \gls{mig} have on the positional measurement using a camera, \gls{ncc} and laser. 

\vspace{5mm}

\noindent The \emph{f}/1.85 aperture was used for the camera. As shown in Fig. \ref{fig:corrected_setup.png} the working surface was approximately \SI{150}{\milli\meter} below the image sensor location inside the camera. The laser beam diameter for this height was approximately \SI{7}{\milli\meter} at the working surface. The experimental design was divided into the following:
\begin{itemize}
    \item Setup 1
    \item Calibration
    \item Testing
    \item Setup 2
    \item Calibration
    \item Testing
\end{itemize}

\section*{To-Do}
\begin{itemize}
    \item \sout{Say here that the aperture size was this and this.}
    \item \sout{Do I say something about the corrected as well as old setup here?}
    \item \sout{What reason should I provide for the old setup?}
    \item Why did I only do x, y movement for the hexapod? Why was z movement not performed?
    \item Why did we only move 0.3 mm?
    \item Because we are moving 0.3 mm, the repeatability error of hexapod with value \SI{\pm0.5}{\micro\meter} can be ignored? How much is the percentage of error? 0.1667\%
    \item Put the quadrant diagram for performed movements here.
    \item Why rotations were not done? My answer because of NCC.
    \item Put research questions for the next chapter.
    \item \sout{Talk about laser diameter used for our experiments.}
    \item Have you taken into account every parameter you wanna talk about?
    \item Explain transformation matrix. 
\end{itemize}
\chapter{Experimental Design}

As mentioned in Table \ref{table:technical_specifications_hexapod}, the hexapod is able to move to a certain user-given coordinate with a precision of \SI{\pm0.5}{\micro\meter} accuracy. Thus, the goals of testing are:
\begin{itemize}
    \item To verify whether the camera, laser module setup along with \gls{ncc} is able to determine this position,
    \item To what accuracy in comparison to hexapod position.
    \item What is the effect of exposure time of the camera on position measurement?
    \item What is the effect of gain of the camera on position measurement?
\end{itemize}

\vspace{5mm}

\noindent In order to achieve the testing goal, the procedure can be divided into the following parts:

\begin{itemize}
    \item \textbf{Implementation of \gls{ncc} algorithm}
    \item \textbf{Calibration}
    \item \textbf{Constant Parameters}: During data collection 
    \item \textbf{Dataset Collection}
    % \item \textbf{Testing Analysis}
\end{itemize}


\section{Implementation of \gls{ncc} algorithm}\label{section:code_logic}
    \subsection*{Template Location}
        The camera image sensor has rectangular dimensions of 728 pixels \times \ 544 pixels as given in Table \ref{table:camera_specs}. The way the camera is mounted for the experiments, it meant that the width of image was more than it's height. It also became necessary going forward, that a square \gls{roi} at the center of this image frame with dimensions of 128 pixels \times \ 128 pixels would be used as a template for \gls{ncc} going forward. The reason is, for position verification of the hexapod, the data collection procedure made it necessary to use the \gls{roi} at the center of the image frame (See Section \ref{section:data_collection}). This is denoted by the square box outlined in black in Fig. \ref{fig:frame_0_rect.png}. The size for \gls{roi} was chosen for the following reasons:
        \begin{itemize}
            \item it is not too small to lose unique information inside it and the template is still tracked to reasonable accuracy,
            \item but also not too large to slow the process of template matching.
        \end{itemize}
        
    \subsection*{Coordinate Axes for Relative Position Measurement}
        In OpenCV, as shown in Fig. \ref{fig:frame_0_rect.png}, for a pixel coordinate (X, Y), X increases as we are going towards the right and Y increases as we traverse through the pixels downwards. For the project, while using \gls{ncc} with OpenCV two functions were employed namely, \texttt{matchTemplate()} and \texttt{minMaxLoc()}. The former function matches the template with the underlying image and saves it's results in another matrix. The latter function allows us to use the result from first function in order to find the location of correlation peak. In doing so, it gives the absolute location of top-left corner of the tracked template according to the OpenCV coordinate axes. Therefore, in order to measure relative movement of the hexapod from the center of image frame, changes were made to reflect pixel shift from origin to center of the \gls{roi}. For e.g., if the hexapod moves from (\SI{0}{\milli\meter}, \SI{0}{\milli\meter}) to (\SI{3}{\milli\meter}, \SI{3}{\milli\meter}) in the first quadrant, the template is tracked in the the fourth quadrant. This is shown by the arrow in Fig. \ref{fig:frame_60_rect.png}. 

        \begin{figure}[h]
            \centering
            \includegraphics[width=0.65\textwidth]{images/e_exp_design/frame_0_rect.png}
            \caption{\gls{lsp} showing OpenCV coordinate axes.}
            \label{fig:frame_0_rect.png}
        \end{figure}
        
        \begin{figure}[h]
            \centering
            \includegraphics[width=0.67\textwidth]{images/e_exp_design/frame_60_rect.png}
            \caption{\gls{lsp} showing relative movement with respect to center of frame.}
            \label{fig:frame_60_rect.png}
        \end{figure}

\clearpage

\vspace{5mm}

\section{Calibration}\label{section:calibration}
    \subsection*{Camera Parameters}\label{subsection:camera_parameters}
        As camera parameters, such as gain and exposure time, have an effect on images being acquired, it is important to decide the values of these parameters for best possible calibration. As discussed in Section \ref{Subsubsection:Exposure_Time}, by increasing/decreasing exposure time one can over-/underexpose an image. This over-/underexposure results in decreasing the average \gls{mig} parameter (discussed in Chapter \ref{section:mig}), because there are less speckles to differentiate between, and hence less change in gradient. On the other hand, more is the variation between speckles inside a template, there is a higher chance of having more uniqueness inside the chosen template, making it easier for \gls{ncc} algorithm to detect pixel shifts. Pixel shifts need to be as accurate as possible, because these values will determine the calibration parameters, as will be discussed in upcoming Section \ref{subsection:calib_matrix}. 
        
        \vspace{5mm}
        \noindent By increasing gain, noise is introduced in the frame, and it too overexposes an image, as intensity values for each of the pixels are increased. In the end, it was decided that exposure time will be decided on basis of highest average \gls{mig} value and gain of the camera will be set to zero, while capturing images for the purpose of calibration.

        \begin{figure}[h]
            \centering
            \begin{subfigure}[b]{0.4\textwidth}
                \centering
                \includegraphics[width=\textwidth]{images/e_exp_design/underexposed.png}
                \caption{Gain: 0, Exposure Time: \SI{20}{\micro\second},\\\gls{mig}: 66.5}
                \label{subfig:underexposed.png}
            \end{subfigure}
            \hspace{1cm}
            \begin{subfigure}[b]{0.4\textwidth}
                \centering
                \includegraphics[width=\textwidth]{images/e_exp_design/overexposed.png}
                \caption{Gain: 0, Exposure Time: \SI{1000}{\micro\second},\\\gls{mig}: 110.0}
                \label{subfig:underexposed.png}
            \end{subfigure}

            \vspace{5mm}
            
            \begin{subfigure}[b]{0.4\textwidth}
                \centering
                \includegraphics[width=\textwidth]{images/e_exp_design/gain_0.png}
                \caption{Gain: 0, Exposure Time: \SI{150}{\micro\second},\\\gls{mig}: 388.85}
                \label{subfig:underexposed.png}
            \end{subfigure}
            \hspace{1cm}
            \begin{subfigure}[b]{0.4\textwidth}
                \centering
                \includegraphics[width=\textwidth]{images/e_exp_design/gain_20.png}
                \caption{Gain: 20, Exposure Time: \SI{150}{\micro\second},\\\gls{mig}: 36.48}
                \label{subfig:underexposed.png}
            \end{subfigure}
            \caption{\glsplural{lsp} for different pairs of gain and exposure time with corresponding \gls{mig} values.}
        \end{figure}

    \subsection*{Image Acquisition Procedure}\label{subsection:image_acq_calib}
        For the two setups outlined in Fig. \ref{fig:old_setup_potrait.JPG} and Fig. \ref{fig:new_setup_potrait.JPG}, image acquisition process involved, that the hexapod was moved \SI{0.1}{\milli\meter} along X- and Y-axis. For the older setup (See Fig. \ref{fig:old_setup_potrait.JPG}), the image acquisition was done only once for both axes. For the corrected setup (See Fig. \ref{fig:new_setup_potrait.JPG}), hexapod was moved 40 times along both axes. The reason being, it allows better estimation of calibration parameters by averaging the influence of pixel-locking effects and vibrations during calibration. As this was a later discovery, because of time concerns the same could not be done for older setup. Fig. \ref{fig:testing.png} shows the direction of movement of hexapod for calibration indicated in purple. The XY axes here correspond to the axes mentioned in Fig. \ref{fig:frame_60_rect.png}.

        \begin{figure}[h]
            \centering
            \includegraphics[width=0.7\textwidth]{images/e_exp_design/testing.png}
            \caption{Image acquisition procedure for testing and calibration. (All coordinates have units in mm. Not to scale.)}
            \label{fig:testing.png}
        \end{figure}


    \subsection*{Calibration Matrix}\label{subsection:calib_matrix}
        As discussed in Section \ref{section:code_logic}, \gls{ncc} gives out detected template's position in terms of pixel coordinates. Therefore, in order to convert pixel shift into displacement, the following method of calibration was adopted for our experiments, as described by Charrett et. al. in their paper \cite{charrett_2018}.

        \vspace{5mm}

        \noindent For calibration, hexapod is moved \SI{0.1}{\milli\meter} along hexapod's X and Y axis at separate times. These distances can be named as, $a_x$ and $a_y$ respectively for both axes. The images are acquired at the same time for these movements. Considering that hexapod moves a certain distance $a_x$ and that there is misalignment between hexapod's and image sensor's XY plane, it would be detected as pixel shift ($A_{xx}$, $A_{yx}$) by the \gls{ncc} algorithm. Here, the first subscript denotes the direction of pixel shift and second subscript denotes direction of movement of hexapod. Similarly, for hexapod's movement along y-axis by distance $a_y$, it would be detected as ($A_{xy}$, $A_{yy}$) in pixel shift. Now, the values for calibration matrix can be calculated as per the formulae given in Eqn. \ref{eqn:calib_matrix_param}. The double script notation allows that, in case of misalignment between the hexapod's XY axes with respect to image sensor plane, for movement of hexapod along one axis, the speckle shift for other axis is accounted for. For e.g., if hexapod moves along x-axis, the speckle shift is accounted for it's x and y axis. This is also illustrated in the following Fig. \ref{fig:misalignment.png}.

        \begin{figure}[h]
            \centering
            \includegraphics[width=0.4\textwidth]{images/e_exp_design/misalignment.png}
            \caption{Schematic showing misalignment between hexapod axes (in black) and image sensor axes (in green).}
            \label{fig:misalignment.png}
        \end{figure}
            
        \vspace{5mm}
        \noindent The pixel shift is related to the actual translation by the following formulae \cite{charrett_2018}:

        \begin{equation}\label{eqn:calib}
            A = T \cdot a
        \end{equation}

        \begin{equation}\label{eqn:calib_matrix}
            \begin{bmatrix}
                A_x \\
                A_y
            \end{bmatrix}
            &=
            \begin{bmatrix}
                T_{xx} & T_{xy} \\
                T_{yx} & T_{yy}
            \end{bmatrix}
            \cdot
            \begin{bmatrix}
                a_x \\
                a_y
            \end{bmatrix}
        \end{equation}

        \noindent with elements of $T$ given by:

        \begin{equation}\label{eqn:calib_matrix_param}
            \begin{aligned}
                T_{xx} = A_{xx} / a_x \\ 
                T_{yx} = A_{yx} / a_x \\
                T_{xy} = A_{xx} / a_y \\
                T_{yy} = A_{yy} / a_y 
            \end{aligned}
        \end{equation}

        \vspace{5mm}
        \noindent Now, if one wants to verify the hexapod distance, we can use the following procedure. Assuming the hexapod is moved to a random coordinate ($a_{x\ actual}$, $a_{y\ actual}$). The images are recorded for such movement at the same time. After these images are analyzed using \gls{ncc} algorithm, one knows the pixel shifts ($A_{xx}$, $A_{yx}$, $A_{xy}$, $A_{yy}$),  as well as the calibration matrix parameters ($T_{xx}$, $T_{yx}$, $T_{xy}$, $T_{yy}$) from calibration procedure performed before. This in the end results, that Eqn. \ref{eqn:dist_calc} can be used to calculate ($a_x$, $a_y$), which can be compared with actual hexapod coordinates ($a_{x\ actual}$, $a_{y\ actual}$). 

        \begin{equation}\label{eqn:dist_calc}
            \begin{aligned}
                a_x = \frac{A_x \cdot T_{yy} - A_y \cdot T_{xy}}{T_{xx} \cdot T_{yy} - T_{xy} \cdot T_{yx}} \\
                a_y = \frac{A_y \cdot T_{xx} - A_x \cdot T_{yx}}{T_{xx} \cdot T_{yy} - T_{xy} \cdot T_{yx}}
            \end{aligned}
        \end{equation}

\section{Constant Parameters}\label{section:constant_parameters}
    \noindent While collecting the data for testing as well as calibration, the following experimental factors are made sure to be constant. The \emph{f}/1.85 aperture was used for the camera while capturing the images. As shown in Fig. \ref{fig:corrected_setup.png} and Fig. \ref{fig:old_setup.png}, the working surface is approximately \SI{150}{\milli\meter} below the image sensor location. The distance between laser collimation point and center of image sensor is approximately \SI{26}{\milli\meter}. The laser beam was collimated approximately at the location of image sensor, resulting in semi-major axis of elliptical laser beam to be approximately \SI{7}{\milli\meter} at the working surface.
    
\section{Data Collection}\label{section:data_collection}
\subsection*{Choosing exposure times for the camera}
As testing the camera's entire exposure range would be impractical, it became necessary to choose specific exposure time values. It can be seen in the Fig. \ref{fig:exposure_time_mig.png}, exposure time $E5=\SI{1000}{\micro\second}$ exhibits similar \gls{mig} value to $E1=\SI{20}{\micro\second}$, the latter being the lowest feasible exposure time for the camera used in our setup. This pairing should demonstrate the effects on position measurement when images are captured with extremely low exposure times, and an exposure time with similar \gls{mig} values. Exposure time $E3=\SI{150}{\micro\second}$ was chosen, because it has remarkably high \gls{mig} value. It also helps in answering, whether increasing contrast in an image has any effect on the accuracy of position measurement. $E2$ is positioned midway between exposure times $E1$ and $E3$. Similarly, $E4$ was chosen for it's similar \gls{mig} value to $E3$, providing additional data point for analysis. $E6=\SI{3000}{\micro\second}$ was chosen to test, how images with extremely low-gradient fare in position measurement.

\begin{figure}[h]
    \centering
    \includegraphics[width=0.7\textwidth]{images/e_exp_design/exposure_time_mig.png}
    \caption{\gls{mig} values for their corresponding exposure times.}
    \label{fig:exposure_time_mig.png}
\end{figure}

\subsection*{Choosing gain values for the camera}
To observe effects of gain alone on position measurement, necessity arose to decide on a single exposure time. Here, $T3$ was chosen, based on it's higher \gls{mig} value. Choosing images with higher contrast i.e. higher \gls{mig} value became apparent, as it felt unnecessary to test effects of gain on position measurement for images captured with extremely low or higher exposure times. The gain for the camera can be changed in whole number increments from $0$ to $40$. But for our experiments, only the following gain values were tested: $0$, $2$, $4$, $6$, $8$, $10$, $20$, and $30$. 

\subsection*{Data Collection Procedure}\label{subsection:data_collection_procedure}
Keeping the testing goals in mind, for each pair of exposure time and gain values of the camera, the hexapod was moved to each of the quadrants 40 times to the specified coordinate, and simultaneously images were captured for each movement of the hexapod (See Fig. \ref{fig:testing.png}). The movement of hexapod to the specified coordinate is named according to the following:

\begin{itemize}
    \item $(0.3, 0.3)$ is \textsf{TopRight}.
    \item $(-0.3, 0.3)$ is \textsf{TopLeft}.
    \item $(-0.3, -0.3)$ is \textsf{BottomLeft}.
    \item $(0.3, -0.3)$ is \textsf{BottomRight}.
\end{itemize}

\noindent The reasoning behind moving an absolute distance of \SI{0.3}{\milli\meter} along XY axes is that, the image sensor having rectangular dimensions, does not allow the greater displacement along Y-axis. In case of greater displacements, template reset is necessary, otherwise the chosen template will move out of the image window, and will not be trackable. As template reset was not implemented for this project, it was decided to move diagonally to each of the coordinates in the four quadrants. The design choice, to have template at the center of the frame, also allowed hexapod to be moved equal distances from the origin (See Fig. \ref{fig:frame_60_rect.png} and Fig. \ref{fig:testing.png}).
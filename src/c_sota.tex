\chapter{State of the Art}\label{chapter:sota}
In this chapter state-of-the-art criteria is covered in four sections. Firstly, experimental setup factors that have an effect on \gls{lsp} is covered in Section \ref{section:exp_factors}. Secondly, the experimental findings of different speckle quality criteria are given a brief overview in Section \ref{section:quality_lsp}. In Section \ref{section:speckle_tracking}, the focus shifts towards feature tracking approaches pertaining to \gls{lsp} images. Finally, Section \ref{section:real_template_matching} explores the real-world applications of template matching. 

% \begin{itemize}
%     \item Experimental setup factors affecting \gls{lsp} (Section \ref{section:exp_factors})
%     \item Different quality criteria for speckle patterns (Section \ref{section:quality_lsp})
%     \item Speckle tracking approaches for \gls{lsp} (Section \ref{section:speckle_tracking})
%     \item Real applications of template matching (Section \ref{section:real_template_matching})
% \end{itemize}

\section{Experimental setup factors affecting \glsfirst{lsp}}\label{section:exp_factors}

\subsection{Camera Parameters}

    \subsubsection{Exposure Time}\label{subsubsection:Exposure_Time}
    Exposure time is defined by the amount of time the digital sensor inside the camera is exposed to light. A decrease in exposure time means, that less light is allowed to fall on the image sensor of a digital camera. Higher frame rates helps in capturing more frames in a given time, consequently image sensor is exposed to light for a shorter amount of time. This can result in darker images, but also captures motions that appear to be frozen in images. Thus, motion blur is minimized. If for the same movement exposure time is increased, motion blur happens, as a result of which speckles would not be observed on the image sensor. Charrett et al.  used an exposure time of \SI{200}{\micro\second} \cite{charrett_2018}. Bandari et. al. used an exposure time of \SI{600}{\micro\second}. No explanation was given by the authors for choosing these values \cite{bandari}.

    \vspace{5mm}

    \noindent Hu et al. confirmed that speckle images captured for exposure values between \emph{B6-B10} gave speckle images of high quality and contrast \noindent \cite{hu}. For their experiment defocussing degree is kept unchanged, and exposure time was adjusted from 1000-16000 resulting in a group 16 speckle images named \emph{Group B: B1-B16} (See Fig. \ref{fig:hu_fig8}). The authors did not provide any units for these values. However, it was concluded that a balance needs to be found between under- or overexposure of image sensor, as any of the extremes would result in a decrease of contrast.

    \begin{figure}[ht]
        \centering
        \includegraphics[width=0.6\textwidth]{images/c_sota/hu_fig8.png}
        \caption{\gls{lsp} captured at different exposure times. (Adapted from Hu et al. 2021) \cite{hu}}
        \label{fig:hu_fig8}
    \end{figure}

    \subsubsection{Frame Rate}\label{Subsubsection:Frame_Rate}
    \Gls{lsp} captured with higher frame rate cameras would exhibit less motion blur owing to more frames being captured in a given time. Charret et al. and Bandari et al. have used 500 \gls{fps} high speed camera in their setup \cite{charrett_2018} \cite{bandari}. In \cite{charrett_mars} authors mentioned that if frame rate is high, then low velocities cannot be measured as the detected translation by correlation algorithms would be less than \~0.1 pixels when using sub-pixel interpolation or 1 pixel otherwise. Therefore, appropriate frame rate should be chosen for required velocity range.

    
    \subsubsection{Aperture Size}
    Aperture is the hole that allows the incoming light to fall on image sensor. Its size controls the exposure of image sensor to light. In the literature concerning speckle patterns, aperture size has an effect on speckle size. Song et al. captured \gls{lsp} for different aperture numbers (See Table \ref{table:song_table_c}) and named them respectively from C1 - C5. As aperture number is increasing, amount of available light falling on sensor decreases, as shown by \gls{agl} values, but \gls{ass} is increasing \cite{song}. Charrett et al. used a `lensless' or \emph{objective speckle setup} with no imaging lens in front of image sensor. Thus, removing the need for aperture to control speckle size \cite{charrett_2018, francis_autonomous}.
    
    \begin{table}[ht]
        \centering
        \footnotesize
        \renewcommand{\arraystretch}{1.2}
        \begin{tabular}{cccc}
            \toprule
            \textbf{Speckle Pattern} & \textbf{Aperture} & \textbf{\gls{ass} (Pixels)} & \textbf{\gls{agl}} \\
            \midrule
            
            C1 & 8 & 2.8 & 188.1 \\
            C2 & 11 & 3.9 & 103.8 \\
            C3 & 16 & 4.9 & 52.1 \\
            C4 & 22 & 7.4 & 25.6 \\
            C5 & 32 & 11.6 & 13.9 \\
    
            \bottomrule
        \end{tabular}
        \caption{\gls{lsp} collected for different aperture sizes (Adapted from Fig. \ref{fig:song_fig9} in Song et al. 2020) \cite{song}.}
        \label{table:song_table_c}
    \end{table}

    \begin{figure}[ht]
        \centering
        \includegraphics[width=0.6\textwidth]{images/c_sota/song_fig9.png}
        \caption{\gls{lsp} collected under different conditions: C1 - C5 for apertures (8, 11, 16, 22, 32). D1 - D5 for different laser powers (10 mW, 100 mW, 200 mW, 300 mW, 400 mW) and E1 - E5 for different temperatures (1200 °C, 1300 °C, 1400 °C, 1500 °C, 1600 °C). (Reprinted from Song et al. 2020) \cite{song}}
        \label{fig:song_fig9}
    \end{figure} 

    % \begin{figure}[h]
    %     \centering
    %     \includegraphics[width=0.7\textwidth]{images/c_sota/song_fig5.png}
    %     \caption{Speckle patterns collected at different apertures and their binary patterns: Speckle patterns A1, A2, A3, A4, A5 with the same \gls{mig} and correspond to apertures 8, 11, 16, 22, 32; Speckle patterns B1, B2, B3, B4, B5 with the same \gls{miosd} and correspond to apertures 8, 11, 16, 22, 32. \cite{song}}
    %     \label{fig:song_fig5}
    % \end{figure}

    
    \subsubsection{Resolution of Captured Image and Template/Subset Size}\label{subsubsection:template_size}
    Charrett et al. used an image with a resolution of \(512 \times 512\) pixels, employing an image sensor with square pixels of edge length \SI{4.8}{\micro\meter} \cite{charrett_2018}. The \gls{ncc} correlation window dimensions were set at \(128 \times 128\) pixels. While a smaller correlation window allows easier \gls{ncc} computation, excessively small windows may increase the likelihood of outliers, thus compromising accuracy. In the study by Pan et al. \cite{pan_subset}, the impact of template size on displacement measurement accuracy was investigated. The authors developed an algorithm adjusting the template size based on the speckle quality parameter \gls{sssig}. For \gls{dic}, an optimal template should contain unique information to ensure reliable displacement measurements. Large templates, capturing more `randomness' are generally preferred, although, with sufficient image contrast, small templates can yield reliable results as well \cite{pan_subset}. Lecompte et al. \cite{lecompte} confirmed that speckle and template size influence the accuracy of displacement measurements. Large templates contribute to more accurate measurements due to increased information content. However, the study primarily shows the influence of speckle and subset size on accuracy without describing a method for choosing a optimal speckle pattern or template size. Yaofeng et al. similarly concluded that large templates lead to more accurate results. But there is an upper limit to large template sizes because of lower performance due to increased computation times \cite{yaofeng}.

    \subsubsection{Speckle Size in Lensed and Lens-less Setup}\label{Subsubsection:Speckle_Size}
    The detected speckle size is largely influenced by the size of the aperture hole. Particle size increases as aperture value (\emph{f-number}) increases as a result of diffraction \cite{song}. Hu et al. stated that optimal speckle size is between 3 - 5 pixels for better quality of \gls{lsp}. If the speckle size is too small, it leads to image under-sampling causing large interpolation bias. On the other hand, if the speckle size is too large, it leads to poor quality of images because of low detail in images, resulting in large random errors \cite{hu}. Lecompte et al. concluded that when using smaller templates, the speckle pattern with large speckles led to most inaccurate results \cite{lecompte}. Charrett et al. used a speckle size of 4 pixels for their objective speckle setup (See Section \ref{Subsubsection:Objective_Subjective}) \cite{charrett_2019}. Francis et. al. suggested a pixel square of 2 \times\ 2, so that each speckle is made of 4 pixels \cite{francis_autonomous}. For an objective and a subjective speckle setup, speckle size (\sigma) can be measured by equation \ref{eqn:objective} and equation \ref{eqn:subjective} respectively \cite{cloud}. Because of an imaging lens in subjective speckle setup, speckle size can change based on the magnification factor ($M$) and $f$-number of the lens. 
    
    \vspace{5mm}
    \begin{itemize}
        \item $\lambda$ is wavelength of laser used
        \item \emph{L} is observation distance
        \item \emph{A} is diameter of illuminated area
        \item \emph{F/a} is the \emph{f}-number of the lens
        \item \emph{M} is the magnification factor
    \end{itemize}

    \begin{equation}\label{eqn:objective}
        \textlangle \sigma_O \textrangle \ = \  \frac{\lambda L}{A}
    \end{equation}

    \begin{equation}\label{eqn:subjective}
        \textlangle \sigma_S \textrangle \ = \  \lambda \frac{F(1 + M)}{a}
    \end{equation}

    \vspace{5mm}

    \noindent Owing to use of lens and an aperture in subjective speckle setup (See Section \ref{Subsubsection:Objective_Subjective}), to increase speckle size, aperture hole size needs to be decreased (See Eqn. \ref{eqn:subjective}). Charrett et al. found that velocity measurement performance increased when speckle size was in between 2 - 5 pixels as seen in Fig. \ref{fig:charrett_mars_fig8} \cite{charrett_mars}.

    \begin{figure}[ht]
        \centering
        \includegraphics[width=0.4\textwidth]{images/c_sota/charrett_mars_fig8.png}
        \caption{Measured velocity versus actual velocity for different speckle sizes. (Adapted from Charrett et al. 2010 \cite{charrett_mars})}
        \label{fig:charrett_mars_fig8}
    \end{figure}

    \subsubsection{Objective versus Subjective Speckle Setup}\label{Subsubsection:Objective_Subjective}

    As shown in Fig. \ref{fig:francis_fig2}, Charrett et al. define ``objective speckle setup" as a setup without a lens and a diaphragm in front of image sensor, while ``subjective speckle setup" contains a lens and diaphragm [Source: Page 3480, Section 2A, \cite{francis_autonomous}]. According to Francis et al. and Charrett et al., it has the following advantages: decreased number of components, and removal of aperture to control speckle size \cite{charrett_2018, francis_autonomous}.

    \begin{figure}[ht]
        \centering
        \includegraphics[width=0.7\textwidth]{images/c_sota/francis_fig2.png}
        \caption{Schematic of (a) Objective and (b) Subjective speckle setup (Adapted from Francis et al. 2012 \cite{francis_autonomous})}
        \label{fig:francis_fig2}
    \end{figure}

    \noindent Francis et al. used a ``subjective speckle setup" for their experiments, where laser beam was diverged to a diameter of \SI{18}{\milli\meter} to extend beyond the \gls{fov} of imaging system. This decreased influence of gaussian beam profile of laser beam, allowing the central part of beam, where intensity is the highest, to extend beyond the imaging area. Although this allows better and consistent illumination, but it decreased the signal levels. The decreased signal levels were further compounded because of smaller aperture size being used in the setup. Hence, slower velocities were used for translation as compared to objective speckle setup \cite{francis_autonomous}.

%     \begin{table}[h]
%         \centering
%         \footnotesize
%         \renewcommand{\arraystretch}{1.2}
%         \begin{tabular}{p{3cm}p{5.5cm}p{5.5cm}}
%             \toprule
%             \textbf{Parameter} & \textbf{Objective Speckle} & \textbf{Subjective Speckle} \\
%             \midrule
%             Signal    & Dependent on configuration. & Dependent of \emph{F}-number of imaging lens. \\
%             Scaling Factor  & Dependent on configuration. & Can be determined from measurement of  \gls{fov}. \\
%             Illumination beam divergence & Relatively small to control speckle size. & Relatively large to overfill \gls{fov}. \\
%             \bottomrule
%         \end{tabular}
%         \caption{Summary of Parameters Pertinent to Choice of Speckle Type for Velocimetry Applications \cite{francis_autonomous}}
%         \label{table:francis_table1}
%     \end{table}
    
%     \begin{figure}[h]
%         \centering
%         \includegraphics[width=0.7\textwidth]{images/c_sota/francis_fig12.png}
%         \caption{The path calculated by cross-correlation of objective speckle patterns with the stages running at a maximum velocity of \SI{80}{\milli\meter/\second} (a). A pair of speckle patterns in successive frames at a point where the stages are moving at maximum velocity (b) and (c), and the normalized cross-correlation between them (d).\cite{francis_autonomous}}
%         \label{fig:francis_fig12}
%     \end{figure}

%     \begin{figure}[h]
%         \centering
%         \includegraphics[width=0.7\textwidth]{images/c_sota/francis_fig13.png}
%         \caption{The path calculated by cross-correlation of subjective speckle patterns with the stages running at a maximum velocity of \SI{50}{\milli\meter/\second} (a). A pair of speckle patterns in successive frames at a point where the stages are moving at maximum velocity (b) and (c), and the normalized cross-correlation between them (d). \cite{francis_autonomous}}
%         \label{fig:francis_fig13}
%     \end{figure}
% \clearpage

\subsection{Impact of Setup Geometry on Speckle Pattern Analysis}

    \subsubsection{Distance between beam waist and detector}\label{Subsubsection:Distance_Beam_Waist_Detector}
    Charrett et al. set the distance between detector (D) and beam waist (S) to be 20mm as seen from Fig. \ref{fig:charrett_2018_fig2} \cite{charrett_2018}. Francis et al. \cite{francis_autonomous} defined scaling factor as ratio between speckle translation to actual translation. The authors stated that sensitivity of setup (indicated by the scaling factor) decreases as angular separation between observation and illumination direction increases. Higher scaling factor is beneficial, as it gives better speckle movement detection for actual small displacements of workpiece \cite{francis_autonomous}.

    \begin{figure}[h]
        \centering
        \includegraphics[width=0.5\textwidth]{images/c_sota/charrett_2018_fig2.png}
        \caption{(a) Schematic of sensor setup (b) 3D printed sensor mount. (Adapted from Charrett et al. 2018 \cite{charrett_2018})}
        \label{fig:charrett_2018_fig2}
    \end{figure}


    \subsubsection{Cone Angle}
    As mentioned in the section \ref{Subsubsection:Distance_Beam_Waist_Detector}, when the SRD angle (See Fig. \ref{fig:charrett_2018_fig2}) is small and the distance between beam waist and detector is small, the incidence and observation angles are closer to the normal. This results, that sensor was robust to changes in working height and small misalignments \cite{charrett_2018}. Francis et al. showed that, the smaller is the angular separation between the incidence angle of laser and observation angle of detector, the better is the sensitivity. Higher sensitivity is beneficial in better speckle movement detection, because a small displacement applied to sensor setup translates to larger displacement in image plane \cite{francis_autonomous}.
    
    
    \subsubsection{Height from the work surface}
    Charrett et al. and Bandari et al. used a setup where beam waist is \SI{150}{\milli\meter} above the working surface (See Fig. \ref{fig:charrett_2018_fig2}) \cite{charrett_2018, bandari}. No explanation was given for the specified value. The authors also mentioned that change in working height resulted in positional error, as that meant that the laser spot translated by a certain value. Francis et al. chose \SI{210}{\milli\meter} \cite{francis_autonomous}. Nagai et al. tested their setup by changing heights between \SI{70}{\milli\meter} to \SI{500}{\milli\meter}, and the laser spot diameter remained at an approximate value of \SI{3.7}{\milli\meter}. The results indicated that error in measurement suddenly increased after a certain value of height, owing to the weak luminance of the reflected laser as seen from Fig. \ref{fig:nagai_fig5}. Apart from this, their findings on changing height during translation of sensor setup in XY-plane indicate that error does accumulate with either increment or decrement of height but remained within 5\% of \SI{200}{\milli\meter} as seen from Fig. \ref{fig:nagai_fig7} \cite{nagai}.

    \begin{figure}[h]
        \centering
        \includegraphics[width=0.5\textwidth]{images/c_sota/nagai_fig5.png}
        \caption{Measurement errors for different height between sensor and surface. 2L and 1L indicate the number of laser sources. (Adapted from Nagai et al. 2010 \cite{nagai})}
        \label{fig:nagai_fig5}
    \end{figure}

    \begin{figure}[h]
        \centering
        \includegraphics[width=0.5\textwidth]{images/c_sota/nagai_fig7.png}
        \caption{Measurement error for height changes during \SI{200}{\milli\meter} movement along Y-axis. (Adapted from Nagai et al. 2010 \cite{nagai})}
        \label{fig:nagai_fig7}
    \end{figure}

    \subsubsection{Tilt/Yaw of the work surface}\label{subsubsection:tilt_yaw}
    Charrett et al. \cite{charrett_2018} explored the impact of tilt and yaw motions in their experimental investigations using the setup in Fig. \ref{fig:charrett_2018_fig4}. Their findings highlighted that tilt in the surface induces more pronounced effects on measurement errors compared to variations in the distance between the camera-laser plane and the workpiece surface. Additionally, the study observed that the sensor's offset from the rotation center results in the measurement of additional velocities. The magnitude of these velocities depends upon both the translation amount and the rotational angle of the surface, in relation to the amount of offset from the center of rotation. Moreover, the relative movement between the sensor setup and the surface introduced spurious velocity measurements due to the surface's movement, other than in-XY-plane translations (for e.g., in-plane-rotations, out-of-plane translations, and tilts). This led to instantaneous changes in velocity measurements occurring only during the duration of this error motion.

    \begin{figure}[h]
        \centering
        \includegraphics[width=0.4\textwidth]{images/c_sota/charrett_2018_fig4.png}
        \caption{Setup used by Charrett et al. for their experiments. (Adapted from Charrett et al. 2018 \cite{charrett_2018})}
        \label{fig:charrett_2018_fig4}
    \end{figure}

    \vspace{5mm}
    \noindent The authors also found that any yaw motion (about the Z-Axis) will result in rotation of speckle pattern and also record additional translation depending upon amount of offset from the center of rotation. It is suggested in this study to have laser spot ideally be colocated at the robot \gls{tcp} \cite{charrett_2018}. Any kind of tilt/yaw affects the efficacy of the \gls{ncc} algorithm negatively, as \gls{ncc} fails to take into account image rotation. This is further confirmed by Charrett et al. in another study \cite{charrett_extended_theory}. They tested surfaces with different gradients (See Fig. \ref{fig:charrett_extended_fig2}) and found out that, it leads to significant errors in correlation. Sjoedahl found that, with a tilt angle of less than 1/\emph{f}, with \emph{f} being the \emph{f}-number of the camera, resulted in an undetectable pattern \cite{sjoedahl}.
    
    \begin{figure}[h]
        \centering
        \includegraphics[width=0.7\textwidth]{images/c_sota/charrett_extended_fig2.png}
        \caption{The 3D printed test objects used by Charrett et. al. in their experiments. $m$ denotes the slope of the surfaces. (Adapted from Charrett et al. 2014 \cite{charrett_extended_theory})}
        \label{fig:charrett_extended_fig2}
    \end{figure}

\subsection{Additional experimental setup factors}

    \subsubsection{Monochromatic and Polychromatic Light}
    Dainty mentions that surface roughness does not affect the speckles statistics when monochromatic light was incident on the surface \cite{dainty}. But the \gls{rms} height variation $\sigma$ did affect the statistics of scattered light when polychromatic light source was used. Apart from the chromatic nature of light source, the coherence of the light also had an effect on the statistical properties of speckle pattern.

    \subsubsection{Velocity}
    Nagai et al. concluded that measurement error increased with increasing speed and acceleration values. But the maximum error was less than 3\% for a velocity of 400 mm/s (See Fig. \ref{fig:nagai_fig8}) \cite{nagai}. Charrett et al. found out that blurring of speckles at high velocities produced positional errors \cite{charrett_2018}. In another study by Charrett et al. \cite{charrett_mars}, low velocities are not measured if the camera's frame rate is high, as the speckle shift detected by correlation algorithms would be less than \~0.1 pixels when using sub-pixel interpolation or 1 pixel otherwise. Smid et al. were unable to record the difference between speckle displacement of 1-3 pixels using \gls{ncc} between frames due to the low velocity values. Minimal speckle displacement recommended by the authors is 4 pixels \cite{smid_2007}.
    
    \subsubsection{Decorrelation}\label{subsubsection:decorrelation}
    Briers et al. mentioned, that when there are small movements to the surface being imaged, the speckles remain correlated. But for larger translations, the speckle patterns \emph{decorrelate} and change completely \cite{briers}. Therefore, \gls{lsp} are sensitive to change in position, yaw, tilt, height. Hu et al. also stated large displacements or tilts should be avoided, as decorrelation would affect the correlation calculation negatively \cite{hu}. Charrett et al. have also mentioned that translation and decorrelation of speckle patterns dependent on object's rotation, translation, strain and surface roughness \cite{charrett_2018}. \Gls{lsp} can deviate also, because of fluctuations in sampling of \Gls{lsp}, random camera noise and power fluctuations in laser source \cite{charrett_2019}.

    % \begin{figure}[h]
    %     \centering
    %     \includegraphics[width=0.5\textwidth]{images/c_sota/nagai_fig10.png}
    %     \caption{Result of sensor output for surface moving at very high speed with sensor placed on plastic plate rotated by turntable. (\cite{nagai})}
    %     \label{fig:nagai_fig10}
    % \end{figure}

    \begin{figure}[ht]
        \centering
        \includegraphics[width=0.5\textwidth]{images/c_sota/nagai_fig8.png}
        \caption{Error at different velocities. (Reprinted from Nagai et. al 2010 \cite{nagai})}
        \label{fig:nagai_fig8}
    \end{figure}
    

    \subsubsection{Laser Power}
    Song et al. collected \glspl{lsp} for different laser powers in Table \ref{table:song_table_d}, and named the corresponding \glspl{lsp} D1-D5 respectively \cite{song}. The laser power affects brightness and contrast of \gls{lsp} as can seen from the \gls{agl} values.

    \begin{table}[ht]
        \centering
        \footnotesize
        \renewcommand{\arraystretch}{1.2}
        \begin{tabular}{cccc}
            \toprule
            \textbf{Speckle Pattern} & \textbf{Laser Power (mW)} & \textbf{\gls{ass} (Pixels)} & \textbf{\gls{agl}} \\
            \midrule
            
            D1 & 10 & 4.0 & 10.4 \\
            D2 & 100 & 3.8 & 83.7 \\
            D3 & 200 & 3.7 & 151.4 \\
            D4 & 300 & 2.9 & 193.4 \\
            D5 & 400 & 1.4 & 222.4 \\
    
            \bottomrule
        \end{tabular}
        \caption{Summary of \gls{lsp} collected for different laser power (Adapted from \ref{fig:song_fig9} in Song et al. 2020 \cite{song}).}
        \label{table:song_table_d}
    \end{table}


    \subsubsection{Temperature}
    Song et al. also sampled \gls{lsp} captured for different surface temperatures, as shown in Table \ref{table:song_table_e} and showed that background radiation in high temperature affects brightness and contrast of speckle pattern, but it does not affect size and shape of speckle particles \cite{song}. This can be observed from the \gls{agl} and \gls{ass} values respectively.

    \begin{table}[ht]
        \centering
        \footnotesize
        \renewcommand{\arraystretch}{1.2}
        \begin{tabular}{cccc}
            \toprule
            \textbf{Speckle Pattern} & \textbf{Temperature (°C)} & \textbf{\gls{ass} (Pixels)} & \textbf{\gls{agl}} \\
            \midrule
            
            E1 & 1200 & 6.8 & 64.3 \\
            E2 & 1300 & 6.7 & 70.8 \\
            E3 & 1400 & 7.0 & 89.5 \\
            E4 & 1500 & 7.3 & 116.7 \\
            E5 & 1600 & 8.6 & 181.1 \\
    
            \bottomrule
        \end{tabular}
        \caption{\gls{lsp} collected under different temperatures (Adapted from Fig. \ref{fig:song_fig9} in Song et al. 2020 \cite{song}).}
        \label{table:song_table_e}
    \end{table}
    
\vspace{5mm}
\subsection{Material Parameters}

    \subsubsection{Surface Roughness}
    Dainty in his article has mentioned that, if a surface has a roughness greater than wavelength of incident light, a speckle pattern will be produced, when a monochromatic laser is incident on it \cite{dainty}. Briers et al. state size of individual speckles is entirely dependent on the wavelength of the incident light and size of the aperture hole used to image the pattern \cite{briers}. Song et al. stated that the smaller the value of roughness is, better is the contrast. But it does not affect the size of speckle particles \cite{song}.
    

    \subsubsection{Surface Material}
    Song et al. collected \glspl{lsp} for different materials at different aperture \emph{f-numbers} (See Table \ref{table:song_table_g}) \cite{song}. The captured \glspl{lsp} corresponded to G1-G5 respectively (See Fig. \ref{fig:song_fig15}). For an aperture size, ceramic has higher \gls{agl} values as compared to stainless steel and carbon-composite. With increasing \emph{f}-number, the \gls{agl} values decrease because of less available light falling on sensor for all materials. Another effect of decreasing aperture hole size is, that \gls{ass} is increasing for all materials.
    \begin{table}[h]
        \centering
        \begin{subtable}{\textwidth}
            \centering
            \footnotesize
            \begin{tabular}{cccccccc}
                \toprule
                \textbf{Speckle Pattern} & \textbf{Aperture ($f$-number)} & \multicolumn{3}{c}{\textbf{\gls{agl}}} \\
                &  & \multicolumn{1}{c}{\textbf{Ceramic}} & \multicolumn{1}{c}{\textbf{Stainless Steel}} & \multicolumn{1}{c}{\textbf{Carbon Composite}} \\
                \midrule
                G1 & 8 & 203.2 & 137.5 & 23.7 \\
                G2 & 11 & 116.7 & 75.2 & 13.7 \\
                G3 & 16 & 60.0 & 38.0 & 8.4 \\
                G4 & 22 & 30 & 21.1 & 5.9 \\
                G5 & 32 & 16 & 11.3 & 4.7 \\
                \bottomrule
            \end{tabular}
        \caption{\gls{agl} values for different apertures and corresponding materials.}
        \end{subtable}
        
        \vspace{3mm}

        \begin{subtable}{\textwidth}
            \centering
            \footnotesize
            \begin{tabular}{cccccccc}
                \toprule
                \textbf{Speckle Pattern} & \textbf{Aperture ($f$-number)} & \multicolumn{3}{c}{\textbf{\gls{ass} (Pixels)}} \\
                &  & \multicolumn{1}{c}{\textbf{Ceramic}} & \multicolumn{1}{c}{\textbf{Stainless Steel}} & \multicolumn{1}{c}{\textbf{Carbon Composite}} \\
                \midrule
                G1 & 8 & 3.0 & 3.6 & 3.6 \\
                G2 & 11 & 4.7 & 3.5 & 3.5 \\
                G3 & 16 & 5.3 & 4.5 & 4.5 \\
                G4 & 22 & 7.5 & 6.6 & 6.6 \\
                G5 & 32 & 12.6 & 8.9 & 8.9 \\
                \bottomrule
            \end{tabular}
        \caption{\gls{ass} values for different apertures and corresponding materials.}
        \end{subtable}

        \caption{Summary of \glspl{lsp} collected for different apertures and materials. (Adapted from Fig. \ref{fig:song_fig15} in Song et al. 2020 \cite{song})}
        \label{table:song_table_g}
    \end{table}

    \begin{figure}[h]
        \centering
        \includegraphics[width=0.65\textwidth]{images/c_sota/song_fig15.png}
        \caption{Materials used and corresponding \glspl{lsp} (a) Different materials (b) \gls{lsp} of different material specimens for diofferent aperture sizes: G1, G2, G3, G4, G5 correspond to apertures 8, 11, 16, 22, 32. (Adapted from Song et al. 2020 \cite{song})}
        \label{fig:song_fig15}
    \end{figure}

    \clearpage

    \vspace{5mm}
    \noindent Nagai et al. tested their setup to understand how different materials, for e.g. artificial lawn, stone, carpet, sand etc. as shown Fig. \ref{fig:nagai_fig12}, affected the tracking performance of \gls{lsp}. Materials such as paper, stone, plastic and aluminum plates are used, and absolute error recorded was less than \SI{8}{\milli\meter} (4\% of \SI{200}{\milli\meter}). The error recorded for artificial lawn was greater than \SI{10}{\milli\meter} (5\% of \SI{200}{\milli\meter}) (See Fig. \ref{fig:nagai_fig11}) \cite{nagai}.


    \begin{figure}[h]
        \centering
        \includegraphics[width=0.45\textwidth]{images/c_sota/nagai_fig12.png}
        \caption{Various materials used in experiment by Nagai et al. (Adapted from Nagai et al. 2010 \cite{nagai})}
        \label{fig:nagai_fig12}
    \end{figure}

    \begin{figure}[h]
        \centering
        \includegraphics[width=0.4\textwidth]{images/c_sota/nagai_fig11.png}
        \caption{Measurement error for \SI{200}{\milli\meter} displacement on different surface materials. 2L and 1L denotes the number of laser sources used. (Adapted from Nagai et al. 2010 \cite{nagai})}
        \label{fig:nagai_fig11}
    \end{figure}

    
\section{Quality criteria for speckle patterns}
    This section covers the experimental findings for tests conducted on speckle patterns using the quality criteria by respective authors. The fundamentals of these quality parameters are covered in Section \ref{section:quality_lsp}.

% \begin{figure}
%     \centering
%     \includegraphics[width=0.6\textwidth]{images/c_sota/image_quality_criteria.jpeg}
%     \caption{Overview of quality criteria covered in literature survey.}
%     \label{fig:image_quality_criteria.jpeg}
% \end{figure}

\subsection{Local Image Quality Criteria}

    \subsubsection{\glsfirst{sssig}}\label{subsubsection:sssig}
        Pan et al. investigated the effect of subset size on measurement accuracy using the parameter \gls{sssig} \cite{pan_subset}. the authors took three different sample images and by changing different subset sizes conducted displacement tests. It was found that as \gls{sssig} increased, standard deviation of displacement error decreased. The authors also`' concluded that displacement measurement accuracy can be improved by the following:
        \begin{itemize}
            \item Increasing the subset \glsentryshort{sssig}.
            \item Using a high bit depth camera (12-bit or 16-bit).
            \item Increasing the contrast of image. The authors did not provide any specifics here.
            \item Decreasing the image noise by using cooled \gls{ccd} and high quality camera lens.
        \end{itemize}

    \subsubsection{Subset Entropy}
        The experiment findings by Yaofeng et al. showed that subset entropy can quantify subset image quality and has direct effect of correlation accuracy. It was observed by authors that standard deviation of measured displacement decreased from $\sim 0.05$ pixels to $\sim 0.02$ pixels, when subset entropy $\delta$ increased from $\sim 0.35$ to $\sim 0.6$ \cite{yaofeng}.

    \subsubsection{\glsfirst{ass}}
        Lecompte et al. investigated effects of subset size and speckle size on accuracy of measured displacements, using image morphology method to calculate speckle size \cite{lecompte}. By repeated erosion and dilation procedures, thresholding method is used to identify a speckle. The authors concluded from the conducted experiments that, larger subsets with larger speckle sizes yielded more accurate results for measured displacements in comparison to small subsets with large speckles. However, this study only demonstrated that there is dependence of speckle size and subset size on displacement measurement, but does not provide any method to obtain optimal speckle size or subset size \cite{lecompte}.

            % \subsubsection{Image Filter Techniques}
            % According to Charrett et al., using image filters such as binary threshold and prewitt kernels (See Fig. \ref{fig:charrett_mars_fig12}) lead to reliable velocity measurements \cite{charrett_mars}. 


            % \begin{figure}[h] 
            %     \centering
            %     \includegraphics[width=0.55\textwidth]{images/c_sota/charrett_mars_fig12.png}
            %     \caption{Typical speckle images: (a) no pre-filter, (b) Canny edge detection, (c) binary threshold, (d) Prewitt gradient filters. \cite{charrett_mars}}
            %     \label{fig:charrett_mars_fig12}
            % \end{figure}

            % \clearpage

            % \begin{figure}[h]
            %     \centering
            %     \includegraphics[width=0.7\textwidth]{images/c_sota/charrett_mars_fig13.png}
            %     \caption{(a) The mean calculated v component of velocity and (b) the minimum correlation Q-value that occurred plotted against stage velocity using the \gls{ncc} method and different pre-filters. \cite{charrett_mars}}
            %     \label{fig:charrett_mars_fig13}
            % \end{figure}


    \subsection{Global Image Quality Criteria}
    
    \subsubsection{\glsfirst{mig}}\label{subsubsection:mig}

        Pan et al. found that \gls{mig} has direct effect on mean bias error and standard deviation of error for measured displacements. The use of global quality parameter is justified by authors, because speckles inside an image are normally distributed. Therefore, the local parameters for different subsets have minimal differences \cite{pan_mig}. It was concluded from experiments that, with increment in \gls{mig}, mean bias error and standard deviation of error decreased. For a subset size of 61 \times\ 61 pixels, a speckle image with \gls{mig} value of 34.6444 demonstrated standard deviation of error of 0.001 pixels for a numerical displacement of 1 pixel. Conversely, speckle image with \gls{mig} value of 9.0379 exhibited maximum standard deviation of error of 0.007 pixels (See Fig. 5 in \cite{pan_mig}).   

        % \begin{equation}
        %     \delta_f = \dfrac{\displaystyle \sum_{i=1}^{W} \displaystyle \sum_{j=1}^{H} |\nabla f(x_{ij})|}{(W \times H)}
        % \end{equation}

        % \noindent where,
        % \begin{itemize}
        %     \item \(|\nabla f(x_{ij})| = \sqrt{f_x(x_{ij})^2 + f_y(x_{ij})^2}\) is the modulus of local intensity gradient
        %     \item \( f_x(x_{ij}), f_y(x_{ij}) \) are first derivatives of intensity along X- and Y-axis respectively at a pixel point $x_{ij}$
        %     \item $W$ and $H$ are the image width and height.
        % \end{itemize}
    
    \subsubsection{\glsfirst{miosd}}
        Yu et al. found that, measurement accuracy is dependent on both \gls{mig} and \gls{miosd}. According to the authors, high quality speckle patterns should have large \gls{mig} and lower \gls{miosd}. This quality parameter reflects the smoothness of gray surface of speckle patterns. Motivation of the authors behind needing this parameter is, that even speckle patterns with same \gls{mig} ($\delta_f$) can have different accuracies for measured displacements. As per the findings from experiments conducted by authors, the so called `good' speckle pattern should have large \gls{mig} and small \gls{miosd}. This correlation can be observed from Fig. \ref{fig:yu_miosd_mig.png} and \ref{fig:yu_miosd_error.png}, where for speckle size than 2 pixels, larger deviations in measured displacement are observed, which correlates to a higher \gls{miosd} value. On the other hand, if speckle size was greater than 5 pixels, image had a small \gls{mig}, which correlates to larger deviations in position measurement \cite{yu_miosd}.

        \begin{figure}[ht] 
            \centering
            \includegraphics[width=0.4\textwidth]{images/c_sota/yu_miosd_mig.png}
            \caption{\glsfirst{mig} and \glsfirst{miosd} for 12 computer generated speckle patterns. (Adapted from Yu et al. 2014 \cite{yu_miosd})}
            \label{fig:yu_miosd_mig.png}
        \end{figure}

        \begin{figure}[ht] 
            \centering
            \includegraphics[width=0.9\textwidth]{images/c_sota/yu_miosd_error.png}
            \caption{Mean bias error and standard deviation of error for numerical displacement of 0.25 pixels (left) and 0.75 pixels (right) respectively for 12 computer generated speckle patterns (Adapted from Yu et al. 2014 \cite{yu_miosd})}
            \label{fig:yu_miosd_error.png}
        \end{figure}
    
        % \begin{equation}
        %     \omega_f = \dfrac{\displaystyle \sum_{i=1}^{W} \displaystyle \sum_{j=1}^{H} |\nabla^2 f(x_{ij})|}{(W \times H)} 
        % \end{equation}

        % \noindent where,
        % \begin{itemize}
        %     \item \(|\nabla^2f(x_{ij})| = \sqrt{f_{xx}(x_{ij})^2 + f_{yy}(x_{ij})^2}\) is modulus of second derivative of local intensity and,
        %     \item $f_{xx}(x_{ij})$ and $f_{yy}(x_{ij})$ are second derivatives of intensities in X- and Y-directions of a pixel $x_{ij}$. 
        % \end{itemize}

        % \noindent The second derivatives can be calculated as follows:

        % \begin{equation}
        %     f_{xx}(x_{ij}) = f(i,j-1) - 2f(i,j) +f(i,j+1)
        % \end{equation}

        % \begin{equation}
        %     f_{yy}(x_{ij}) = f(i-1,j) - 2f(i,j) + f(i+1,j)
        % \end{equation}

    \subsubsection{\glsfirst{msf}}
        Hua et al. applied numerical translations on the speckle patterns and observed that mean bias error in measured displacements is small when \gls{msf} value ($S_f$) is high \cite{hua_msf}. For a numerical displacement of 1 pixels applied on the speckle images, maximum mean bias error of 0.06 pixels was observed for a speckle image with \gls{msf} value of 64.51. Conversely, for speckle image with \gls{msf} value of 114.51, showed maximum mean bias error of 0.01 pixels.  Optimal speckle size between 2 to 4 pixels was suggested by the authors, as it displayed higher \gls{msf} values compared to other speckle sizes. Thus, a correlation was drawn between influence of speckle size on position measurement accuracy. 

        % \begin{figure}[h]
        %     \centering
        %     \includegraphics[width=0.35\textwidth]{images/c_sota/msf_subset.png}
        %     \caption{Selecting points inside a subset. (Adapted from Hua et al. 2011 \cite{hua_msf})}
        %     \label{fig:msf_subset.png}
        % \end{figure}        
        
        % \begin{equation}
        %     S_p = \sum_{i=1}^{3} \sum_{j=1}^{3} |a_{ij} - \overline{a}|
        %     \label{eqn:local_msf}
        % \end{equation}

        % \begin{equation}
        %     S_f = \dfrac{\sum_{p \in F} S_p}{H \times V}
        %     \label{eqn:global_msf}
        % \end{equation}
    
    \subsubsection{\glsfirst{se}}
        % Liu et al. developed a quality parameter based on concept of entropy. This criterion quantifies information content of an image and captures `randomness' inside a speckle pattern.

        % \begin{equation}
        %     H(Y) = - \displaystyle \sum_{j=0}^{2^{\beta} - 1} p(a_j)\ log(p(a_j)) 
        % \end{equation}

        % where,
        % \begin{itemize}
        %     \item $H$ represents Shannon entropy (bits/pixel), 
        %     \item $\beta$ denotes pixel depth of the image,
        %     \item $p(a_j)$ signifies normalized probability of occurrence of each gray level. 
        % \end{itemize}

        Liu et al. numerically deformed the images to avoid errors introduced by imaging system. The experimental findings showed that images with large \gls{se} had small mean bias error. This trend can be observed from Fig. \ref{subfig:liu_shannon_entropy.png}, where for speckle patterns with high \gls{se} values, corresponding mean error values are low. Another conclusion drawn by authors was, that continuously increasing number of speckles, does not always correlate to higher accuracy. High \gls{se} values denotes that speckle image has more feature information or higher speckle `uniqueness' \cite{liu_shannon}.

        \begin{figure}[ht] 
            \centering
            \begin{subfigure}{0.49\textwidth}
                \centering
                \includegraphics[width=\textwidth]{images/c_sota/liu_shannon_entropy.png}
                \caption{\glsfirst{se} values.}
                \label{subfig:liu_shannon_entropy.png}
            \end{subfigure}
            \begin{subfigure}{0.49\textwidth}
                \centering
                \includegraphics[width=\textwidth]{images/c_sota/liu_mean_error.png}
                \caption{Mean error values versus speckle numbers.}
                \label{subfig:liu_mean_error.png}
            \end{subfigure}
            \caption{\glsfirst{se} and mean error values for 23 computer generated speckle patterns. (Reprinted from Liu et al. 2015 \cite{liu_shannon})}
            \label{fig:liu_shannon}
        \end{figure}

    \subsubsection{\glsfirst{sdgis}}
        The motivation of Park et al. behind using this parameter is that, using gray-intensity alone to judge the speckle pattern quality does not take into account the variability of speckles within the image. The authors generated 12 artificial speckle patterns for this study using airbrush techniques, and found that higher \glsentryshort{sdgis} meant higher level of gray-gradient and hence provided more `uniqueness' in image correlation. In Fig. \ref{fig:park_sdgis_error.png}, it can be clearly observed that as \gls{sdgis} value ($\rho$) is increasing, mean error in pixels is decreasing. The correlation coefficient between this error and $\rho$ is -0.8793 with a \emph{p}-value of 1.9977 \times\ $10^{-16}$. The legend inside this figure shows the values 0.2, 0.5, 1.0, and \SI{1.2}{\milli\meter} which correspond to different nozzle sizes used for airbrushing, whereas white, gray and black colors indicate the low-, medium- and high-volume fractions respectively which is addressed by varying spraying time \cite{park_sdgis}.

        \begin{figure}[ht] 
            \centering
            \includegraphics[width=0.5\textwidth]{images/c_sota/park_sdgis_error.png}
            \caption{Relationship between $\rho$ and mean error for 12 different speckle patterns. (Adapted from Park et al. 2017 \cite{park_sdgis})}
            \label{fig:park_sdgis_error.png}
        \end{figure}
            
        
        % \vspace{5mm}
        % \noindent Quantifying quality of a speckle pattern involves, firstly, identifying boundary of speckles by converting a speckle pattern image to a binary image using Otsu's method of thresholding \cite{otsu}. \glsentryshort{sdgis} calculation is then made for each speckle. As this is still local assessment of quality for only one speckle, the general quality assessment is done for the subset region. Distribution curves of \glsentryshort{sdgis} are generated for each subset, and then averaged over a given percentage for overall quality assessment. More details can be found in their study \cite{park_sdgis}.

        % \begin{equation}
        %     \rho = SD_{R50} \dfrac{SD_{R75} - SD_{R25}}{0.5}
        % \end{equation}

        % \noindent where,
        % \begin{itemize}
        %     \item $\rho$ is speckle pattern quality,
        %     \item $SD_{R25}$, $SD_{R50}$, $SD_{R75}$ are lower quartile (25\%), median (50\%) and upper quartile (75\%) in cumulative distribution curve of \glsentryshort{sdgis}.
        % \end{itemize} 

    \subsubsection{E\textbf{_f}}
        According to Hu et al. \cite{hu_ef}, most of the existing literature on speckle pattern quality uses only one parameter to evaluate mean bias error and standard deviation of error. On the contrary, the principle of both these error models are quite different from each other. $E_f$ was also compared with different image quality criteria by using 8 artificial speckle patterns. The translation movement was applied on speckle patterns numerically using the fourier shifting method \cite{schreier}. In conclusion, the authors found that order of quality of speckle patterns can differ from one criteria to another because of different definitions of quality and different care indexes. With regards to \gls{mig} and $E_f$, the order of speckle pattern quality are aligns with experimental findings. Additionally the authors observed, that in relation to mean bias error and standard deviation of error, $E_f$ and \gls{mig} are more efficient in assessing the speckle pattern quality as compared to other image quality parameters like \gls{se} and \gls{mffi} \cite{hu_ef}.

        % \vspace{5mm}
        % \noindent The bias error calculation $k_x$, $k_y$ was done for both X- and Y-directions for whole speckle pattern using the Eqn. \ref{eqn:ef_kx} and \ref{eqn:ef_ky}. It can be seen from these equations that $E_f$ is related to first and second order image intensity gradient. 
        
        % \begin{equation}
        %     k_x = \dfrac{ \sum_{i=1}^{W} \sum_{j=1}^{H} |\nabla^2 f_x (x_{ij}) \ \nabla f_x (x_{ij}) | }{ \sum_{i=1}^{W} \sum_{j=1}^{H} [\nabla f_x (x_{ij})]^2 }
        %     \label{eqn:ef_kx}
        % \end{equation}
        
        % \begin{equation}
        %     k_y = \dfrac{ \sum_{i=1}^{W} \sum_{j=1}^{H} |\nabla^2 f_y (x_{ij}) \ \nabla f_y (x_{ij}) | }{ \sum_{i=1}^{W} \sum_{j=1}^{H} [\nabla f_y (x_{ij})]^2 }
        %     \label{eqn:ef_ky}
        % \end{equation}

        % \begin{equation}
        %     E_f = \sqrt{k{_x}^{2} + k{_y}^{2}}
        % \end{equation}

    \subsection{Miscellaneous}

    \subsubsection{\glsfirst{mffi}}

        Song et al. mentioned that most of the traditional quality criteria consider either contrast of the speckle pattern or speckle morphology and are evaluated for artificial speckle patterns instead of \gls{lsp} \cite{song}. To take speckle morphology and gray level information together into account, the authors developed \gls{mffi} quality criteria ($\delta_{MFFI}$). According to findings from the authors, the smaller $\delta_{MFFI}$ is better is the \gls{lsp} quality, which correlated to lower mean bias error and standard deviation of error in measured displacements \cite{song}.
        
        % To quantify quality of a \gls{lsp}, Song et al. suggested a new parameter called \gls{mffi}. It takes 3 things into account:
        % \begin{enumerate}
            
        %     \item \textbf{\gls{igd} ($\delta_{IGD}$)}: If the pixels inside a \gls{lsp} are distributed across all possible gray levels $L$, $\delta_{IGD}$ will have the smaller value in comparison to an image where all pixels have same gray intensity values. According to authors, gray levels being distributed means, speckle patterns contains a lot of `information' \cite{song}.
                
        %         \begin{equation}
        %             \delta_{IGD} = \sqrt{\sum_{k=0}^{L-1} P_{k}^{2} / (W \times H)}
        %         \end{equation}
                
        %         \noindent where, 
        %         \begin{itemize}
        %             \item $P_k$ denotes the number of pixels of a particular gray level inside a \gls{lsp},
        %             \item $W$ and $H$ denote the width and height of image in pixels,
        %             \item $L$ denotes gray level number of an image.
        %         \end{itemize}

        %         \noindent If $L = 5$, then $k = {0, 1, 2, 3, 4}$ and number of pixels inside each gray level are given by \(\{P_k, k = 0, 1, ... , L-1\}\).

        %     \item \textbf{\gls{msdg} ($\delta_{MSDG}$)}: This parameter quantifies the contrast inside an image. It is defined by Eqn. \ref{eqn:msdg}. It identifies the concentration of a certain gray level distribution. The more dispersed the distribution, the higher the value of this parameter, and the better the quality of speckle pattern \cite{song}. 
                
        %         \begin{equation}
        %             \delta_{MSDG} = \sqrt{\sum_{k=0}^{L-1} \biggl\{ (k - \overline{P})^2 \times Q_k \biggr\}}
        %             \label{eqn:msdg}
        %         \end{equation}

        %         \noindent where,
        %         \begin{itemize}
        %             \item \(Q_k = P_{k}/(W \times H)\)
        %             \item Mean gray value \(\overline{P} = \sum_{k=0}^{L-1} k \times Q_k\)
        %         \end{itemize}   

            
        %     \item \textbf{\gls{sdsps} ($\delta_{SDSPS}$)}: This parameter is calculated by Eqn. \ref{eqn:sdsps}, and considers the speckle morphology inside a \gls{lsp}. This parameter gives a high value when differences between speckle sizes is large. The number of speckles inside a speckle pattern is calculated firstly by Otsu's thresholding \cite{otsu}, and then boundary detection to calculate number of points inside each speckle \cite{song}.  

        %     \begin{equation}
        %         \delta_{SDSPS} = \sqrt{\frac{1}{T} \displaystyle \sum_{n=1}^{T} \  (A_n - \overline{A})^2} (max(A_n) > 1)
        %         \label{eqn:sdsps}
        %     \end{equation}

        %     where,
        %     \begin{itemize}
        %         \item $A_n$ denotes size of every speckle particle,
        %         \item mean speckle size \(\overline{A} = \sum_{n=1}^{T} A_n / T\),
        %         \item $T$ is number of speckle particles.
        %     \end{itemize} 
        % \end{enumerate}

        % \noindent Finally, these parameters are combined to give $\delta_{MFFI}$ using Eqn. \ref{eqn:mffi}. The authors mentioned, that by combining multiple methods, the shortcoming of considering only a single factor is overcome.

        % \begin{equation}
        %     \delta_{MFFI} = \dfrac{a \times \delta_{IGD}}{\delta_{MSDG} \times \delta_{SDSPS}}
        %     \label{eqn:mffi}
        % \end{equation}

        % \vspace{3mm}
        % \noindent where, $a$ denotes a linear coefficient to adjust value of $\delta_{MFFI}$ for better illustration. 


    \subsubsection{Quality assessment using Deep-Learning based \glsentryshort{cnn}}

        According to the Kwon et al., the current speckle pattern quality measurements are based on preexisting qualitative knowledge of speckle-pattern-features. The authors found from their experiments that using \gls{cnn} for assessing speckle pattern quality provided better performance than existing measures such as \gls{mig}, \gls{msf}, \gls{se}, and \gls{sdgis}. The authors also used different sizes of artificial speckle pattern images, and observed that \gls{cnn} based error prediction demonstrated higher correlation coefficient than the other image quality criteria. The convolutional layer, because of its local fields and shared weights, was able to extract local features. These local features were combined by the layers of \gls{cnn} to calculate global correlations. This permitted local as well as global assessment of quality of speckle patterns \cite{kwon_cnn}.
    
        \begin{table}[h]
            \centering
            \renewcommand{\arraystretch}{1.5}
            \footnotesize
            \begin{tabular}{p{2.25cm}p{1cm}p{5cm}p{5cm}p{1cm}}
                \toprule

                \textbf{Parameter} & \textbf{Scope} & \textbf{Speckle Pattern Type} & \textbf{Displacement Test Method} & \textbf{Year} \\
                
                \midrule

                \glsentryshort{mig}\cite{pan_mig} & Global & Various techniques involved: Spray painting, black marker dots, metal surface polishing using sandpaper & Numerically deformed by applying shift in Fourier domain using shift theorem & 2010 \\

                \midrule

                \glsentryshort{miosd}\cite{yu_miosd} & Global & Computer generated speckle patterns & Numerically deformed by applying shift in Fourier domain using shift theorem & 2014 \\

                \midrule

                \glsentryshort{msf}\cite{hua_msf} & Global & Artificial speckle patterns by randomly spraying white and black paint & Numerical translations applied by using shift theorem & 2011 \\

                \midrule

                \glsentryshort{se}\cite{liu_shannon} & Global & Numerically generated speckle patterns, artificial speckle patterns by spraying white and black paint & Strain testing of specimen by applying tension upto \SI{2}{\kilo\newton} at \SI{0.25}{\kilo\newton} intervals & 2013 \\

                \midrule

                \glsentryshort{sdgis}\cite{park_sdgis} & Global & Artificial speckle patterns by airbrush technique & Numerical deformation of speckle patterns to perform following tests:
                \begin{itemize}[leftmargin=*]
                    \item Uniaxial compression
                    \item 3-point bending
                    \item Mode-I crack
                \end{itemize} & 2017 \\

                \midrule

                \( E_f \)\cite{hu_ef} & Global & Artificial speckle patterns by airbrush techniques and black marker dots & Numerical deformation of images by using fourier shifting method & 2021 \\

                \midrule

                \glsentryshort{sssig}\cite{bomarito} & Local & Artificial speckle pattern by airbrushing black paint, numerically generated speckle patterns & Three tests conducted:
                \begin{itemize}[leftmargin=*]
                    \item Test 1: Numerical deformation of numerically generated speckle using fourier interpolation
                    \item Test 2: Actual speckle patterns and numerical deformations
                    \item Test 3: Actual speckle patterns and actual displacements of surface relative to camera
                \end{itemize} & 2016 \\

                \midrule

                Subset Entropy\cite{yaofeng} & Local & Speckle patterns captured under white light & Rigid body translation by moving the positioning stage by a few \SI{}{\micro\meter} (no specific value provided) & 2007 \\ 

                \midrule

                \glsentryshort{ass}\cite{lecompte} & Local & Spray painting to create black speckles over a white surface & Speckle pattern numerically deformed using finite element simulated displacement field & 2006 \\

                \midrule

                \glsentryshort{mffi}\cite{song} & Misc. & \gls{lsp} & \gls{lsp} numerically shifted in fourier domain & 2020 \\

                \midrule

                \glsentryshort{cnn}\cite{kwon_cnn} & Misc. & Artificial speckle patterns using airbrush and spray painting & Numerically displacement of images to perform following tests:
                \begin{itemize}[leftmargin=*]
                    \item Uniaxial compression
                    \item 3-point bending
                    \item Mode-I crack
                \end{itemize} & 2023 \\
                
                \bottomrule
            \end{tabular}
            \caption{Summary of speckle pattern quality criteria.}
            \label{table:summary_quality_criteria_1}
        \end{table}


        \begin{table}[h]
            \centering
            \footnotesize
            \begin{tabular}{m{2.2cm}m{6.25cm}m{6.25cm}}
                \toprule

                \textbf{Parameter} & \textbf{Pros} & \textbf{Cons} \\
                
                \midrule
                %%%%%
                
                \glsentryshort{mig}\cite{pan_mig} & 
                \begin{itemize}[leftmargin=*]
                    \item Widely used parameters for speckle quality measurement \cite{hu_ef}
                \end{itemize}
                 & 
                \begin{itemize}[leftmargin=*]
                    \item Involves mean operation which ignores variability across the speckle pattern \cite{crammond}
                    \item Speckle morphology and gray level distribution is ignored \cite{song}
                \end{itemize} \\ 
                \midrule
                %%%%%
                
                \glsentryshort{miosd}\cite{yu_miosd} & 
                \begin{itemize}[leftmargin=*]
                    \item  Takes gray information of speckle image into account
                \end{itemize}
                 & 
                \begin{itemize}[leftmargin=*]
                    \item Ignores speckle morphology \cite{song}
                \end{itemize} \\ 
                \midrule
                %%%%%
                
                \glsentryshort{msf}\cite{hua_msf} & 
                \begin{itemize}[leftmargin=*]
                    \item Provides quantification of contrast
                \end{itemize}
                 & 
                \begin{itemize}[leftmargin=*]
                    \item Involves mean operation which ignores variation within a speckle pattern \cite{crammond}
                    \item Cannot distinguish between speckle size \cite{crammond}
                \end{itemize} \\ 
                \midrule
                %%%%%
                
                \glsentryshort{se}\cite{liu_shannon} & 
                \begin{itemize}[leftmargin=*]
                    \item Quantifies `randomness' or `uniqueness' within a speckle pattern
                \end{itemize}
                 & 
                \begin{itemize}[leftmargin=*]
                    \item Not efficient to assess speckle pattern quality from perspective of mean bias error and standard deviation error \cite{hu_ef}
                \end{itemize} \\ 
                \midrule
                %%%%%
                
                \glsentryshort{sdgis}\cite{park_sdgis} & 
                \begin{itemize}[leftmargin=*]
                    \item  Measures variation of gray intensity within each speckle
                \end{itemize}
                 & 
                \begin{itemize}[leftmargin=*]
                    \item Not suitable for \gls{lsp}, because of small differences in gray std. deviation between individual speckles \cite{song}
                \end{itemize} \\ 
                \midrule
                %%%%%
                
                \( E_f \)\cite{hu_ef} & 
                \begin{itemize}[leftmargin=*]
                    \item Combines multiple factors
                \end{itemize}
                 & 
                \begin{itemize}[leftmargin=*]
                    \item Extreme cases of speckle sizes and density still left unexplored
                \end{itemize} \\ 
                \midrule
                %%%%%

                \glsentryshort{sssig}\cite{bomarito} & 
                \begin{itemize}[leftmargin=*]
                    \item Takes gray information and speckle morphology into account
                \end{itemize}
                 & 
                \begin{itemize}[leftmargin=*]
                    \item Involves mean operation which ignores variability across the speckle pattern \cite{crammond}
                    \item Cannot be used to measure whole speckle pattern quality \cite{song}
                    \item Can be approximated by \gls{mig} for a large subset \cite{pan_mig}
                \end{itemize} \\
                
                \midrule
                %%%%%
                
                Subset Entropy\cite{yaofeng} & 
                \begin{itemize}[leftmargin=*]
                    \item Restricted in their approach to evaluate overall quality of speckle image, variation among subsets is ignored \cite{kwon_cnn}
                \end{itemize}
                 & 
                \begin{itemize}[leftmargin=*]
                    \item Choice of 8 neighboring pixels based on intuitive idea and lacks mathematical support \cite{pan_subset}
                \end{itemize} \\ 

                \midrule
                %%%%%

                \glsentryshort{ass}\cite{lecompte} & 
                \begin{itemize}[leftmargin=*]
                    \item Takes into account influence of speckle size and subset size on accuracy of measured displacements \cite{song}
                \end{itemize}
                 & 
                \begin{itemize}[leftmargin=*]
                    \item Threshold value to identify speckles is decided by user. Hence, threshold method is inaccurate \cite{bomarito}
                    \item Influence of speckle gray information is ignored \cite{song}
                \end{itemize} \\ 

                \midrule
                %%%%%

                \glsentryshort{mffi}\cite{song} & 
                \begin{itemize}[leftmargin=*]
                    \item  Takes multiple local and global factors into account
                \end{itemize}
                 & 
                \begin{itemize}[leftmargin=*]
                    \item Not efficient to assess speckle pattern quality from perspective of mean bias error and standard deviation error \cite{hu_ef}
                \end{itemize} \\ 
                \midrule
                %%%%%

                \glsentryshort{cnn}\cite{kwon_cnn} & 
                \begin{itemize}[leftmargin=*]
                    \item  Combines local and global quality assessment
                \end{itemize}
                 & 
                \begin{itemize}[leftmargin=*]
                    \item Dataset preparation for training
                    \item Prediction accuracy dropped sharply outside training data range \cite{kwon_cnn}
                \end{itemize} \\ 
                
                \bottomrule
            \end{tabular}
            \caption{Pros and Cons of different image quality criteria.}
        \end{table}


\clearpage
\section{Speckle tracking approaches for \gls{lsp}}\label{section:speckle_tracking}

For the purpose of \gls{lsi} with methods other than template matching (See Chapter \ref{chap:fundamentals} Section \ref{section:template_matching}), Charrett et al. investigated feature tracking methods in Ref. \cite{charrett_2019}. For that, the authors collected different datasets of translation and rotation using an aluminum plate of dimensions \SI{200}{\milli\meter} \times\ \SI{200}{\milli\meter}. Datasets were prepared with translations between \SI{0}{\milli\meter} to \SI{2.8}{\milli\meter} in \SI{50}{\micro\meter} steps, and rotations from 0° to 360° in 5° steps. They found that \gls{ncc} along with sub-pixel interpolation algorithm such as 3-point gaussian fit \cite{raffel} had a translation accuracy similar to feature tracking approaches with error/pixel-locking of \sim0.033 pixels for \gls{ncc}, \sim0.04 pixels for \glsentryshort{brisk}\cite{brisk_feature} and \sim0.025 pixels for \glsentryshort{surf}\cite{surf_feature} and \glsentryshort{usurf}\cite{surf_feature} methods. The \glsentryshort{orb}\cite{orb_feature} method had the lowest performance out of the other feature tracking approaches with peak error of 0.17 pixels, because of integer pixel location of key points in the method instead of sub-pixel locations used in other methods (See Fig. \ref{fig:charrett_2019_fig4}). The authors found that feature tracking approaches have an advantage over correlation approaches in measuring rotation. Out of the other feature tracking methods, \glsentryshort{orb} performed the best with reliable measurements of rotation upto \pm25° (See Fig. \ref{fig:charrett_2019_fig5}(a)) \cite{charrett_2019}.


\begin{figure}[h]
    \centering
    \includegraphics[width=0.4\textwidth]{images/c_sota/charrett_2019_fig4.png}
    \caption{Comparison of error in translation measurements between 2D-\gls{ncc} and feature tracking. (Reprinted from Charrett et al. 2019 \cite{charrett_2019})}
    \label{fig:charrett_2019_fig4}
\end{figure} 

\begin{figure}[h]
    \centering
    \includegraphics[width=0.8\textwidth]{images/c_sota/charrett_2019_fig5.png}
    \caption{Evaluation of rotational measurement accuracy by feature tracking methods using rotated dataset of \gls{lsp} (See Table 3 of Ref. \cite{charrett_2019}). (a) Measured rotation versus actual rotation. (b) The number of matched features after application of distance threshold, described in Section 4.2 and Table 2 of Ref. \cite{charrett_2019}. Plots (c,d) show measurement errors, with in (d) vertical scale is changed to outline accuracy clearly. (Reprinted from Charrett et al. 2019\cite{charrett_2019})}
    \label{fig:charrett_2019_fig5}
\end{figure} 

% \vspace{5mm}
% \noindent So far, no algorithms has been found that measures pure rotation. But, that is out of the scope of this thesis and hence will not be tested. \cite{charrett_mars} also compared \gls{ncc} with Radon Transformation and Optical Flow algorithms and concluded them unsuitable for the application of velocity measurement. Radon transformation was deemed unsuitable because it is computationally expensive and Optical Flow approach required calibration for each type of surface to be able to perform velocity measurement. They found that \gls{ncc} was best suited approach with odometry error \~{}0.2\% at speeds of \SI{0.1}{\milli\meter/\second} and 0.75\% at \SI{50}{\milli\meter/\second}. In the methodology used for velocity measurements by Charrett et. al. \cite{charrett_wpos}, because of peak locking \cite{raffel}, Gaussian 3-point interpolation is required to have sub-pixel accuracy to measure translation. Still with each re-referencing procedure used in their methodology, a positional error will accumulate. This can make a difference in the end, if the application is for long-range translation measurement or short range, vibration measurement.

% \section{Summary}\label{section:summary}
% There are a total of 17 factors that can influence the \gls{lsp} and the \gls{dic} techniques as per the literature review. They are as follows:

% \begin{itemize}
%     \item Exposure Time
%     \item Aperture Size
%     \item Frame Rate of Camera
%     \item Laser wavelength
%     \item Correlation Window Size
%     \item Image Resolution of Camera
%     \item Speckle Size
%     \item Objective vs. Subjective Speckle Setup
%     \item Height of Surface from Sensor Setup
%     \item Tilt or yaw for the surface materialto \SI{100}{\micro\meter} resolution (\SI{100}{\milli\meter} working range) at kHz data rates.
%     \item Image Filters in \cite{opencv}
% \end{itemize}

% Methodology:

% \begin{itemize}
%     \item How do different materials affects speckle size and speckle pattern in general?
%     \item charrett2018 500fps, 200us exposure time, 512x512 pixels
%     \item Exposure time? Affected by aperture size and shutter speed. How do these factors affect speckle pattern?
%     \item charrett2018 detector and focus point in same plane and 150mm above the surface
%     \item charrett2018 balanced angle geometry: distance between detector and focus point is 50mm, zero sensitivity to out of plane motion
%     \item charrett2018 effect of the SDR angle on sensitivity
%     \item charrett2018 Objective setup. No lens in front of imaging camera. Therefore small aperture not needed to increase speckle size. So in such a case how does one increase speckle size?
%     \item balance high frame rate with short exposure time. But then increase the camera aperture hole at risk of decreasing pixel size. The short exposure time can be countered with increasing the laser power, working with higher reflectivity material, decreasing the distance between detector and surface, and improving sensitivity of detector \cite{charret_autonomous}
%     \item Therefore the setup needs to be recreatable. Because sensitive to yaw, tilt etc.
% \end{itemize}

\section{Real Applications of Template Matching} \label{section:real_template_matching}

    \subsection{Velocity measurement using \gls{lsp} for a Mars exploration rover}
        In order to improve the odometry information coming from wheel encoders, Charrett et al. explored the possibility of using laser speckle images to perform velocity measurement in this study \cite{charrett_mars}. This study aimed at exploring different algorithms aiding velocimetry and the possibility of forming speckle images using Mars soil analogues using a laboratory setup (See Fig. \ref{fig:charrett_mars_fig2.png}). Out of all the tested algorithms, \gls{ncc} performed the best with odometry error of 0.2\% at speeds of upto \SI{1}{\milli\meter/\second} and less than 0.75\% at \SI{50}{\milli\meter/\second}. They also mentioned that this technique should scale well to higher velocities of \sim\SI{85}{\milli\meter/\second} (usual speed of exploration rovers), if higher frame rate camera is used and better image signals can be captured from the surface. But, testing speckle velocimetry at these speeds requires further investigation. Other open issues concerning this study includes finding out influence of vehicle motions and rotations \cite{charrett_mars}.

        \begin{figure}[ht]
            \centering
            \includegraphics[width=0.3\textwidth]{images/c_sota/charrett_mars_fig2.png}
            \caption{Schematic of laboratory setup used by Charrett et al.\ for their experiments. (Adapted from Charrett et al. 2010 \cite{charrett_mars})}
            \label{fig:charrett_mars_fig2.png}
        \end{figure}
        
    \subsection{Robotic tool speed measurement using \gls{lsp}} \label{subsection:robotic_tool_speed}
        In \cite{charrett_2018}, Charrett et al. proposed a sensor concept that can be used for measurement of robot \gls{tcp} speed. The sensor concept can be seen in Fig. \ref{fig:sensor_concept.png}. It involved using a camera-laser setup which records \gls{lsp} for in-plane movements, and then uses 2D-\gls{ncc} for measuring velocity. Initial experiments are done on a laboratory setup, where relative movement between sensor setup and surface is done using a 6-axis hexapod (See Fig. \ref{fig:charrett_2018_fig4}). The authors reported a high sensor accuracy with maximum error of \pm\SI{0.01}{\milli\meter/\second} over a velocity range of \pm\SI{70}{\milli\meter/\second}. For lower velocity range of \pm\SI{10}{\milli\meter/\second}, error in velocity measurement reduced to \textless \SI{0.004}{\milli\meter/\second}. The sensor's precision was measured by giving hexapod stage circular path, and varying the X and Y velocity components. Here, peak velocity errors \SI{0.34}{\milli\meter/\second} at \SI{1}{\milli\meter/\second} and \pm\SI{2.28}{\milli\meter/\second} at \pm\SI{70}{\milli\meter/\second} were observed \cite{charrett_2018}.
        

    \subsection{Three-\gls{dof} sensor for robot manufacturing}
        Charrett et al. developed another sensor concept (Fig. \ref{fig:charrett_wpos_fig1.png}), for measurement of 3-translation \gls{dof} of robot \gls{tcp} called \gls{wpos} \cite{charrett_wpos}. It involves an optical sensor setup that utilizes the principles of \gls{rri} \cite{kissinger_rri} for out-of-plane positioning, and laser-speckle correlation for in-plane positioning or potential rotation \cite{charrett_2019}. A brief overview of these techniques is provided below:

        \subsubsection{\glsfirst{rri}}
            \gls{rri} is an interferometry technique based on sinusoidal wavelength modulation of laser diodes. It gives high resolution measurement of interferometric phase thus giving nanometer level precision in displacement measurement at multi-kHz bandwidths \cite{kissinger_rri}. In \gls{wpos}, \gls{rri} is used to measure the distance between workpiece surface and robot \gls{tcp}. As per the authors, \gls{rri} also offers advantages such as large working ranges (\geq\ \SI{10}{\centi\meter}), small sensor heads (\leq\ \SI{10}{\milli\meter}), and that interferometric nature of this process allows it to be immune to background light (for example, from arc-welding) \cite{kissinger_rri}.

        \subsubsection{\glsfirst{lsc}}
            This is similar to sensor concept used by Charrett et al. in Ref. \cite{charrett_2018} explained in Section \ref{subsection:robotic_tool_speed}. Here, as well to measure 2D in-plane movement, \gls{ncc} was utilized on \gls{lsp} \cite{charrett_wpos}.

        \vspace{5mm}
        \noindent Finally, the two techniques are compiled here, to form the \glsfirst{wpos}. The authors used this sensor concept for three application areas.
        
        \begin{itemize}

            \item \textbf{Tool Speed Measurement:} Here, \gls{lsc} correlation was used. The performance of the camera-laser sensor setup was investigated in their other work \cite{charrett_2018}. The authors found sensor to have a high accuracy with maximum error of \pm\SI{0.01}{\milli\meter/\second} over a velocity range of \pm\SI{70}{\milli\meter/\second} \cite{charrett_wpos}.
            
            \item \textbf{In-process layer height measurement:} Here, \gls{rri} was used to measure the distance between workpiece and the surface. According to the authors, this is applicable for measurements of layer height during additive manufacturing methods such as \gls{waam} \cite{hallam_waam}. The authors also said, that the measurements were made in presence of arc-light, thus proving the immunity of \gls{rri} to background light. It was found that, sensor setup had a resolution between \sim \SI{10}{\micro\meter} to \SI{100}{\micro\meter} for a working range of \SI{100}{\milli\meter}. Out-of-plane vibration testing was also performed, where resolution of \SI{1}{\nano\meter} to \SI{10}{\nano\meter} was achieved \cite{charrett_wpos}. 

            \item \textbf{Real-Time Positioning:} For in-plane positioning authors used \gls{ncc}. In order to perform long range relative position measurements, re-referencing operation is necessary \cite{charrett_wpos}. Re-referencing involves resetting the reference image being used for template tracking, once the translation becomes greater than dimensions of the camera chip used. As any tilts/misalignments can decrease the accuracy of position measurement \cite{charrett_2018}, the authors want to use \gls{rri} to update scaling factors that convert speckle shift to actual distance \cite{charrett_wpos}. Experimental findings conducted show, that for short range measurements errors in position amounted to \textless\SI{0.5}{\micro\meter}, and for long range measurements of \sim \SI{0.5}{\meter}, the error was \sim\SI{120}{\micro\meter}.

        \end{itemize}
            
        \begin{figure}
            \centering
            \includegraphics[width=0.6\textwidth]{images/c_sota/charrett_wpos_fig1.png}
            \caption{\gls{wpos} concept (Adapted from Charrett et al. 2019 \cite{charrett_wpos})}
            \label{fig:charrett_wpos_fig1.png}
        \end{figure}
        
        % \begin{table}[h]
        %     \centering
        %     \footnotesize
        %     \begin{tabular}{c{4cm}c{6cm}}
        %         \toprule
        %         \textbf{Measurement Mode} & \textbf{Notes and Performance} \\
        %         \midrule
                
        %         In-plane tool speed & 
        %         \begin{itemize}[leftmargin=*]
        %             \item 2D velocity measurement via laser speckle correlation. 
        %             \item \pm\SI{0.01}{\milli\meter/\second} over a velocity range of \pm\SI{70}{\milli\meter/\second}.
        %         \end{itemize} \\

        %         In-plane positioning (long range \sim\SI{1}{\meter}) & 
        %         \begin{itemize}[leftmargin=*]
        %             \item Laser speckle correlation (+ working distance correction via \gls{rri})
        %             \item \sim\SI{120}{\micro\meter/\meter} 
        %         \end{itemize} \\
                
        %         In-plane positioning/displacement/vibration (short range \sim mm to cm) & 
        %         \begin{itemize}[leftmargin=*]
        %             \item Laser speckle correlation (+ working distance correction via \gls{rri})
        %             \item Errors \textless \SI{0.5}{\micro\meter} upto \sim10kHz.
        %         \end{itemize} \\
                
        %         \bottomrule
        %     \end{tabular}
        %     \caption{Summary of }
        % \end{table}

\clearpage
\chapter{Conclusion}
    To conclude, this research aimed at choosing \gls{mig} as a \gls{lsp} quality parameter, and its efficacy was tested to verify position measurement using template matching with 2D-\gls{ncc}. Experimental parameters such as exposure time, gain and laser spot diameter were changed, and their clear effect on appearance of \gls{lsp} and \gls{mig} were observed. Based on the conducted tests, there is a correlation between \gls{mig} values and accuracy of position measurement. However \gls{mig} is an incomplete parameter, because the nature of formation of \gls{lsp} is dependent on multiple factors, and it only considers gray level values of the \gls{lsp}, ignoring the other factors such as speckle size and density which have an effect on the accuracy. Therefore, further research is necessary into other speckle quality parameters to address these shortcomings. This research provides a new insight into practical application of a widely used image quality criteria in predicting accuracy of position measurement using \gls{lsp} images, whereas the current state-of-the-art primarily focussed on artificial speckle patterns and numerical deformation of speckle pattern images.

% To conclude, \gls{lsp} analysis using \gls{ncc} algorithm seems a promising alternative to measure relative position change. For small distances, such as the ones used in this project, the positional errors in XY plane have an absolute magnitude of less than \SI{10}{\micro\meter}. It is also to be noted that, camera and setup parameters have an end-effect on appearance of \gls{lsp}. Hence, an image quality parameter becomes a necessity, because the experimental variables are subject to change depending on the requirements. For this project, the chosen \gls{lsp} quality criteria - \gls{mig} - shows that reliable positional accuracy is possible for small distances, regardless of value of gain and exposure time of the camera, provided it's value lies between $\sim66$ and $\sim409$. Additionally, camera's orientation with respect to hexapod surface and laser diameter also plays a significant role in accuracy of position measurement. 

% \vspace{5mm}
% \noindent Currently the makeup of error $e_{total}$ based on the observations can be summarized in Eqn. \ref{eqn:makeup_error} where:

% \begin{itemize}
%     \item $e_{pixel\ locking}$ is the error due to pixel locking and vibrational effects explained in Section \ref{section:results_discussion_calib}. 
%     \item $e_{calibration\ matrix}$ is the error introduced in position calculation due to calibration errors because of pixel locking and vibration.
%     \item $e_{hexapod}$ is the repeatability error in position of hexapod. This amounts to 0.1667\% \cite{hexapod_manual}.
% \end{itemize}
% \begin{equation}
%     e_{total} \approx f(e_{pixel\ locking}, e_{calibration\ matrix}, e_{hexapod})
%     \label{eqn:makeup_error}
% \end{equation}
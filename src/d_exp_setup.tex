\chapter{Experimental Setup}

The experiments were conducted using the following components:

\section*{Camera}
    The camera used for the experiments is manufactured by Allied Vision and the model's name is Mako G-040B. It is a \gls{poe} enabled, Gig-E compliant monochrome camera. It can run at frame rate of 286 \gls{fps} at full resolution of 0.4 megapixel. It has a progressive scan \gls{cmos} Sony IMX287 image sensor with a pixel size of \SI{6.9}{\micro\meter} x \SI{6.9}{\micro\meter}. The rest of the technical details can be found in the Table \ref{table:camera_specs}.

    \begin{table}[h]
        \centering
        \footnotesize
        \renewcommand{\arraystretch}{1.2}
        \begin{tabular}{p{6cm}p{7cm}}
            \toprule
            \textbf{Parameter} & \textbf{Description} \\
            \midrule
            Interface & IEEE 802.3 1000BASE-T, IEEE 802.3af (\gls{poe})\\
            Resolution & 728 (H) x 544 (V)\\
            Sensor & Sony IMX287\\
            Sensor Type & \gls{cmos}\\
            Shutter Mode & \gls{gs}\\
            Sensor Size & Type 1/2.9\\
            Pixel Size & \SI{6.9}{\micro\meter} x \SI{6.9}{\micro\meter}\\
            Lens Mounts & C-Mount, CS-Mount\\
            Max. Frame Rate at Full Resolution & 286 \gls{fps}\\
            ADC & 12 Bit\\
            Image Buffer & 64 MByte\\
            \bottomrule
        \end{tabular}
        \caption{Specifications for Mako G-040B Camera. \cite{mako_camera}}
        \label{table:camera_specs}
    \end{table}

    \begin{figure}[h]
        \centering
        \includegraphics[width=0.5\textwidth]{images/d_exp_setup/mako_camera.png}
        \caption{Mako G-040 \cite{mako_camera}}
        \label{fig:mako_camera.png}
    \end{figure}

\clearpage

\section*{Laser Diode Module}
    The laser module CPS635F from ThorLabs (See Fig. \ref{fig:laser_module.jpg}) has typical wavelength of \SI{635}{\nano\meter}. It has adjustable focal length and produces laser beam of elliptical shape. The variations in output wavelength with changing temperature can be seen from the Fig. \ref{fig:spectrum_laser_module.png}. Other specifications can be found in the Table \ref{table:laser_module_specs}.

    \begin{figure}[h]
        \centering
        \includegraphics[width=0.26\textwidth]{images/d_exp_setup/laser_module.jpg}
        \caption{CPS635F Laser Module from ThorLabs. \cite{thorlabs_laser}}
        \label{fig:laser_module.jpg}
    \end{figure}

    \begin{figure}[h]
        \centering
        \includegraphics[width=0.5\textwidth]{images/d_exp_setup/spectrum_laser_module.png}
        \caption{Spectrum of CPS635F module taken at 20\textdegree{}C, 25\textdegree{}C, 30\textdegree{}C. This data is typical and will vary for each module. \cite{thorlabs_laser}}
        \label{fig:spectrum_laser_module.png}
    \end{figure}

    \begin{table}[h]
        \centering
        \footnotesize
        \renewcommand{\arraystretch}{1.2}
        \begin{tabular}{p{6cm}p{2cm}p{2cm}p{2cm}}
            \toprule
            \textbf{Parameter} & \textbf{Minimum} & \textbf{Typical} & \textbf{Maximum} \\
            \midrule
            Wavelength & \SI{630}{\nano\meter} & \SI{635}{\nano\meter} & \SI{645}{\nano\meter} \\
            Power & \SI{4.0}{\milli\watt} & \SI{4.5}{\milli\watt} & \SI{5.0}{\milli\watt} \\
            Power Stability (8 hours) & - & - & 2\% \\
            Power Stability (1 minute) & - & - & 1\% \\
            Axis Deviation & - & - & \SI{5}{\milli\radian} \\
            Beam Divergence (Collimated) & - & - & \SI{1.6}{\milli\radian} \\
            Focal Range (From Exit Window) & \SI{80}{\milli\meter} & - & Collimated \\
            Focussed Spot Diameter (\SI{100}{\milli\meter}, 1/e²) & - & \SI{30}{\micro\meter} & - \\ 
            Operating Current & - & \SI{50}{\milli\ampere} & \SI{70}{\milli\ampere} \\
            \bottomrule
        \end{tabular}
        \caption{Optical and Electrical Specifications of CPS635F Laser Module. \cite{thorlabs_laser}}
        \label{table:laser_module_specs}
    \end{table}

\clearpage

\section*{Lens Body}
    The lens body holds camera lens as well as the bandpass filter. 
    \subsection*{Camera Lens}
    The camera lens, manufactured by Edmund Optics, has a fixed focal length of \SI{25}{\milli\meter}. The aperture is adjustable from \emph{f}/1.85 to \emph{f}/16. This allows the working distance to be in range \SI{200}{\milli\meter} - $\infty$.

    \subsection*{Bandpass Filter}
    The bandpass filter FLH635-10 from ThorLabs allows 85\% transmission of laser with wavelength of \SI{635}{\nano\meter}. The pass region for this filter is \SI{10}{\nano\meter} at \gls{fwhm}. The bandpass filter is needed to avoid influence of light other than the laser source on the conducted experiments.

    \begin{figure}[h]
        \begin{subfigure}{0.5\textwidth}
            \centering
            \includegraphics[width=0.5\textwidth]{images/d_exp_setup/lens.jpg}
            \caption{Camera Lens. \cite{edmund_optics_lens}}
            \label{fig:lens.jpg}
        \end{subfigure}
        \begin{subfigure}{0.5\textwidth}
            \centering
            \includegraphics[width=0.4\textwidth]{images/d_exp_setup/bandpass_filter.jpg}
            \caption{Bandpass Filter. \cite{thorlabs_bandpass_filter}}
            \label{fig:bandpass_filter.jpg}
        \end{subfigure}
        \caption{Lens Body Components.}
        \label{fig:lens_body.jpg}
    \end{figure}

\section*{Hexapod}
The M-824 hexapod from \gls{pi} contains six linear actuators between base plate and platform. Some of the advantages of this hexapod include it's high stiffness, six-dimensional motion, and high resolution. All motion commands for the movement of the platform are defined by three linear and three rotational coordinate values. The rotational axes defined by (U, V, W) are initially coincident with translational axes (X, Y, Z) of the hexapod coordinate system, as shown in Fig. \ref{fig:hexapod_coordinate.png}. This hexapod is capable of moving \SI{\pm22.5}{\milli\meter} along X and Y axes with a resolution of \SI{0.3}{\micro\meter}. For larger movements in range of few millimeters, the final position values can differ by \SI{\pm0.5}{\micro\meter}. Rest of the technical details can be found in the Table \ref{table:technical_specifications_hexapod}.

\begin{figure}[h]
    \begin{subfigure}{0.5\textwidth}
        \centering
        \includegraphics[width=0.9\textwidth]{images/d_exp_setup/hexapod.jpg}
        \caption{Image of actual hexapod. Citation??}
        \label{fig:hexapod.jpg}
    \end{subfigure}
    \begin{subfigure}{0.5\textwidth}
        \centering
        \includegraphics[width=0.7\textwidth]{images/d_exp_setup/hexapod_coordinate.png}
        \caption{Hexapod coordinate system indicating it's translation axes (X, Y, Z) and rotational axes (U, V, W). Citation??}
        \label{fig:hexapod_coordinate.png}
    \end{subfigure}
    \caption{M-824 Hexapod}
\end{figure}

\begin{table}[h]
    \centering
    \footnotesize
    \renewcommand{\arraystretch}{1.2}
    \begin{tabular}{p{7cm}p{3cm}}
        \toprule
        \textbf{Parameter} & \textbf{Description} \\
        \midrule
        Travel X, Y & \SI{\pm22.5}{\milli\meter} \\        
        Travel Z & \SI{\pm12.5}{\milli\meter} \\        
        Travel $\theta_x$, $\theta_y$ & \SI{\pm7.5}{\degree} \\        
        Travel $\theta_z$ & \SI{\pm12.5}{\degree} \\        
        Min. incremental motion X, Y, Z & \SI{0.3}{\micro\meter} \\
        Min. incremental motion $\theta_x$, $\theta_y$, $\theta_z$ & \SI{3.5}{\micro\radian} \\
        Repeatability X, Y & \SI{\pm0.5}{\micro\meter} \\
        Repeatability $\theta_x$, $\theta_y$, $\theta_z$ & \SI{\pm6}{\micro\radian} \\
        Velocity X, Y, Z (typical) & \SI{0.5}{\milli\meter/\second} \\
        Velocity X, Y, Z (maximum) & \SI{1}{\milli\meter/\second} \\
        Velocity $\theta_x$, $\theta_y$, $\theta_z$ (typical) & \SI{5.5}{\milli\radian/\second} \\
        Velocity $\theta_x$, $\theta_y$, $\theta_z$ (maximum) & \SI{11}{\milli\radian/\second} \\
        Load capacity (Baseplate horizontal) & \SI{10}{\kilo\gram} \\
        \bottomrule
    \end{tabular}
    \caption{Technical specifications of M-824 Hexapod. Citation??}
    \label{table:technical_specifications_hexapod}
\end{table}

\clearpage

\section*{To-Do}
\begin{itemize}

    \item Circuit Design Photo
    \item 

\end{itemize}
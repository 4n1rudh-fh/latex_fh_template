\chapter*{Abstract}

Due to positional inaccuracies in robot \gls{tcp} position, an optical position estimation method utilizing \gls{lsp} images is investigated from the perspective of quantitative quality analysis. In contrast to the prevalent literature which predominantly analyzes quality of artificial speckle patterns created through techniques like airbrushing or spray-painting, this study focusses on \gls{lsp} images. The effectiveness of a popular speckle quality criteria called \gls{mig} in predicting accuracy of measured displacements via 2D-\gls{ncc} is investigated in this study. Experiments are conducted on \gls{lsp} images captured during actual translations on a 6-\gls{dof} hexapod. For the applied translations, maximum positional measurement error of \SI{8}{\micro\meter} were observed in this study. Additionally, the findings suggest that there is a correlation between \gls{mig} values and displacement measurement accuracy. However, \gls{mig}'s predictive accuracy varies under different experimental conditions, due to the intricate nature of \gls{lsp} formation. These results indicate that, while \gls{mig} is a valuable quality criteria, its ability to predict measurement accuracy is limited, emphasizing the need for further research into more comprehensive or adaptable \gls{lsp} quality criteria.   

\vspace{5mm}
\noindent \emph{Keywords:} Laser speckle pattern, mean intensity gradient, speckle pattern quality, displacement, robotic sensor

% \vspace{5mm}
% \noindent The measurement accuracy of a laser-speckle based velocity sensor was analyzed from perspective of \gls{lsp} image quality. Majority of current literature focusses on assessing quality of artificial speckle patterns i.e. speckle patterns formed by using airbrush or spray painting techniques, instead of \gls{lsp}. Based on the literature survey, a widely used criteria called \gls{mig} was chosen to assess quality of \gls{lsp}. In this study, its efficiency to measure accurate displacements using 2D-\gls{ncc} was investigated. Actual translations were performed using a 6-\gls{dof} hexapod, and simultaneously \gls{lsp} images were captured. The results show that although there is a correlation between \gls{mig} and accuracy of measured displacement, \gls{mig} is not efficient in predicting measurement accuracy for all conditions, due to formation of \gls{lsp} dependent on multiple factors.  
\chapter*{Abstract}

Industry 4.0 or the fourth industrial revolution has outlined the objectives, such as automated decision making, interconnected machinery and data analytics to increase productivity in manufacturing sectors. This productivity increase is made possible, especially with the help of robots. But, robots suffer from inherent positional accuracies, and as a result can give sub-optimal outcomes. This cannot be ignored in areas, where precision manufacturing is of higher importance. To counter this issue, the author developes a sensor concept involving a camera and a laser module, which using \gls{ncc} helps determine relative positional change. The purpose of this thesis is, to develop this sensor concept and determine how accurate it is in comparison to a robot. The major research questions for this project are as follows - \emph{What is the effect of camera parameters - exposure time and gain - on position measurement?} \emph{How can one define image quality of a \gls{lsp}?} \emph{What effect does experimental setup parameter - laser diameter - have on position measurement?} To test this, experimental setup was designed to mimic relative movement between robot \gls{tcp} and a given surface within a controlled setup. The idea is, that by integrating this sensor, the issue of positional inaccuracies within a robot will be addressed. Findings from the conducted research show promising results, where positional errors amount to less than \SI{10}{\micro\meter}, and an image quality parameter was identified, which remains independent of experimental variables and is used to assess the quality of a \gls{lsp}. In the future this sensor concept can find it's use in a wide range of industries, where precise robotic functions are of necessity.
\chapter{Introduction}\label{Chap:Introduction}

In many areas of manufacturing it is advantageous to use robots because of their flexibility and low cost efficiency, but they suffer from the problems of low stiffness, low positioning accuracy and insufficient rigidity \cite{ji}, \cite{chen}. Materials that robots are made out of, expand with heat, and hence result in non-linear error. Apart from the given reasons, errors resulting from improper calibration, sensor inaccuracies, and incorrect servo loops \cite{greenway}, \cite{torgny} can result in deviations in robot \Gls{tcp} pose from the intended path as shown in Figure \ref{fig:fig_walderich}. Hence, research has to be conducted in area of minimizing these errors and estimating better robot \Gls{tcp} pose for more accurate machining processes.

\vspace{5mm}

% \noindent Change the following image.

    \vspace{5mm}
    \begin{figure}[h]
        \centering
        \includegraphics[width=0.9\textwidth]{images/a_introduction/Pathaccuracy.JPG}
        \caption{Attained path of a Universal Robot UR5e compared to command path (left) and the resulting specimen during laser cutting (right). (\textcopyright \ Philipp Walderich) \cite{img_walderich}}
        \label{fig:fig_walderich}
    \end{figure}
    \vspace{5mm}

    \noindent In the recent years the prices of lasers have decreased by 70\% and low cost fiber lasers have seen a surge in market demand \cite{optech}. If the aforementioned deviations in robot pose are minimized, robots owing to their advantages of flexibility, low-cost operation along with cheaper lasers can be a viable option in future of \gls{lmp}.
    
    \vspace{5mm}

    \noindent One way to counter these errors is, to have a visual calibration of robot pose from an external sensor. This external sensor, provided it has better accuracy than the robot, can then provide information on the degree of deviation of robot \gls{tcp} path from a given path. Attempts have been done before to improve this positional accuracy using machine vision techniques including photogrammetry, stereo vision, laser tracker systems, and laser triangulation to name a few. 
    
    \vspace{5mm}
    
    
    \noindent Techniques like photogrammetry and stereo vision require better illumination conditions, as these techniques rely on recognizing keypoints and features within the image frame for better accuracy. As a result, physical marks such as reflective stickers or laser points become necessary for these these techniques, as shown in Fig. \ref{fig:fig_perez_4}. Solutions to deal with robot pose involving photogrammetry involves using multiple cameras or attaching \gls{led} reference targets for camera imaging, and reflectors suitable for laser tracking to robot's \gls{tcp}. All of these then should be calibrated with local coordinates to improve accuracy. Another solution is to use a laser tracker, which can track the geometric and dynamic parameters of the robot, as can be seen in Fig. \ref{fig:perez_fig13}. But this kind of setup increases costs, and can suffer from problem of occlusion. If one wants to avoid using physical markers, markerless stereo vision systems involve using feature tracking algorithms to find points of interest within the image frame. These have found their use in indoor and outdoor robot odometry applications. But, these techniques fail, when the surfaces are low-textured \cite{perez}. It is also highly likely that in \Gls{lmp}, one works with surfaces that are featureless. Hence, techniques need to be identified which improve robot \gls{tcp} pose using images without any features.

    \vspace{5mm}
    \begin{figure}[h]
        \centering
        \includegraphics[width=0.7\textwidth]{images/a_introduction/perez_fig4.png}
        \caption{Physical marks used in marker-based stereo vision. (a) Stickers (b) Laser points. \cite{perez}}
        \label{fig:fig_perez_4}
    \end{figure}
    \vspace{5mm}
    
    \noindent Other 3D vision techniques for example, \gls{tof} cameras employ light pulses. Depth information of the scene is calculated based on the time it takes for the reflected light to be received by the sensor. Although \gls{tof} does not get affected by ambient light in the scene, occlusion from other objects can become a problem. Other problems of \gls{tof} as mentioned by Perez et. al. include, low raw data quality because of noise, depth inhomogeneity at object boundaries, multiple reflections from objects \cite{perez}. Moreover, point cloud data generated from these techniques also needs to be processed using algorithms such as \gls{icp}. Hence, the robustness of final setup depends on: point cloud acquisition and its subsequent treatment. The minimum accuracy mentioned by Perez et. al. for \gls{tof} is \SI{10}{\milli\meter} \cite{perez}.
    
    \vspace{5mm}
    \noindent In conclusion, stereo vision and photogrammetry are able to determine robot pose accurately to range of \SI{64}{\micro\meter} \cite{perez}. But it involves processing a large amount of image data, can be influenced by factors such as brightness and lights in industrial environments and solutions to overcome their disadvantages involve costlier approaches. \gls{tof} because of it's noisy data, and low accuracy, is not suitable for the purpose of \gls{lmp}.


    \vspace{5mm}
    \begin{figure}[h]
        % \hspace{3cm}
        \centering
        \includegraphics[width=0.4\textwidth]{images/a_introduction/perez_fig13.png}
        \caption{Laser tracker. \cite{perez}}
        \label{fig:perez_fig13}
    \end{figure}
    \vspace{5mm}

    \noindent \gls{lsi} has been gaining traction in the recent years because of advancements made in camera and signal processing technology \cite{filter},\ \cite{farsad}, and has simpler image processing requirements in comparison to other machine vision techniques. Another one of the advantages of \gls{lsi} is, that it works on surfaces that are featureless \cite{francis_autonomous}. Feature tracking approaches can also be used in \gls{lsi}, because of \gls{lsp} property of producing unique keypoints and patterns even for featureless surfaces. As a result, it has found it's application in industry, vehicle odometry measurements \cite{charrett_mars}, for sound detection and regeneration of audio signals \cite{nan_wu}, and for recognizing solid metal components based on their "laser speckle fingerprint" \cite{sjoedahl}. For the use case of \gls{lmp}, occlusion of laser light for \gls{lsi} is not a problem, as is the case with other machine vision techniques mentioned before. This is because, in a typical scenario, no object should interfere between robot \gls{tcp} and workpiece surface during a \gls{lmp} process. Apart from that, \gls{lsp}'s shape and size is also not affected by high temperatures \cite{song}, and \gls{dic} using \gls{lsp} does not require database of images for correlation \cite{farsad}. The sensor setup for \Gls{lsi} being low-cost is another potential advantage \cite{charrett_2018}.

    \vspace{5mm}
    \noindent The motivation of the project is, to develop a \gls{lsi} sensor, that is more accurate than a typical robot, such that deviations in robot path can be recognized. In the future, these deviations can be worked upon by using control engineering methods to get a more accurate robot pose.
    
\end{fraunhofertext}
\chapter{Discussion}\label{chapter:discussion}

\section{Analysis on Calibration Matrix}

    The results indicate that choice of calibration matrix has a clear influence on final position measurements. Fig. \ref{fig:mean_error_calib_matrices} shows that, $T_3$ produces highest amount of mean error for motions \textsf{BottomLeft} and \textsf{TopRight} along X-axis, compared to other motions. However its measurement accuracy outperforms $T_{Avg}$ across \textsf{BottomRight} and \textsf{TopLeft} by only \SI{1}{\micro\meter}. A similar trend is also observed for $T_2$, where it produces higher errors than $T_{Avg}$ across all motions along Y-Axis. Along, X-axis, $T_2$ outperforms $T_{Avg}$ across \textsf{BottomRight} and \textsf{TopLeft} by only \SI{1}{\micro\meter}. Out of all the calibration matrices, $T_{Avg}$ shows lowest errors in \textsf{BottomLeft} and \textsf{TopRight}, and produces only slightly higher errors in comparison to other matrix counterparts, as demonstrated above. Across all movements, $T_{Avg}$ produced maximum mean errors magnitude of \SI{5}{\micro\meter} along X-axis and \SI{4}{\micro\meter} along Y-axis with standard deviation of < \SI{1}{\micro\meter} across both axes.

    \vspace{5mm}
    \noindent The experimentation conducted here further illustrates the necessity to implement sub-pixel interpolation algorithms to provide better measurement accuracy. Notably, as the dataset was not changed during the experiments, \gls{mig} remained consistent across the different calibration matrices. It is beyond the scope of this study to address the question of whether pixel-locking effects observed in measured pixel displacements are due to the discrete nature of template matching with images or the vibrations encountered during recording the \gls{lsp} images.
    
% Because of the fluctuation of pixel shifts, the each of the parameters of calibration matrix $T$ can have 2 choices. Thus, resulting 16 different permutations of $T$. Instead of testing each of those, 4 were chosen on random basis. Apart from this, another $T$ was calculated based on average of fluctuations across 40 calibrations for each axis (See Table \ref{table:stats_x} and \ref{table:stats_y}). These 5 calibration matrices are given in Eqn. \ref{eqn:calib_matrices_permutations}. From Eqn. \ref{eqn:dist_calc}, the position $(a_x, a_y)$ is decided by the elements inside $T$. Because the matrices differ from each other, so will the final distances that are calculated. For the exposure time of \SI{150}{\micro\second} and gain equal to $0$, it can be seen by comparing Tables from \ref{table:stats_matrix_topright} to \ref{table:stats_matrix_bottomright} that across all movements of hexapod, $T_{Avg}$ fares better in comparison to other matrices. The maximum magnitude of error in position observed for $T_{Avg}$ is \SI{9}{\micro\meter} along X-Axis and \SI{7}{\micro\meter} along Y-Axis (See Table \ref{subtable:t_avg}). \gls{mig} does not change here, because same dataset was used to perform \gls{ncc}.

\section{Analysis on Exposure Time}\label{section:analysis_exposure_time}
    For \gls{lsp} images captured with exposure time of \SI{3000}{\micro\second}, as seen in Fig. \ref{fig:example_images_exposure_time}, it is clear that image contains less identifiable speckle features due to overexposure as compared to other images. This is also evident in the \gls{mig} value which is comparatively low (< 7 units) as compared to other exposure times, due to finding less `edges' inside speckle images. Consequently, this has an effect on position measurement using template matching. From Fig. \ref{fig:mean_error_exposure_time} and \ref{fig:stddev_error_exposure_time}, it can be clearly observed that for exposure time of \SI{3000}{\micro\second}, mean error and standard deviation of mean error increase as compared to other exposure times. \gls{mig} also helps in quantifying contrast in the image, as it has low values for images that are under- or overexposed as compared to images captured with exposure time of \SI{150}{\micro\second}. The experimental findings for this test suggest that, when \gls{mig} lies between \sim66 to \sim390, the position measurements are reliable with maximum mean error of \SI{8}{\micro\meter} with standard deviation of error of \SI{2}{\micro\meter}.

    % \vspace{5mm}
    % \noindent Furthermore, when experimental setup is changed, such as laser module or the camera is switched, or distances between components is changed, it has an effect on camera parameters. Thus, \gls{mig} can be monitored as an independent parameter in such situations. The experimental findings suggest that when \gls{mig} is within a range of $\sim$66 to $\sim$409, the position values are reliable with a maximum absolute error of less than \SI{10}{\micro\meter}.

    % \vspace{5mm}

    % \begin{figure}[h]
    %     \centering
    %     \includegraphics[angle=90, width=0.35\textwidth]{images/f_results_discussion/laser_spot_block.JPG}
    %     \caption{Laser partly incident on lens body.}
    %     \label{fig:laser_spot_block.JPG}
    % \end{figure}

\section{Analysis on Gain}
    Increasing gain has similar effects on appearance of \gls{lsp}, with images captured with gain of 20 and 30 are clearly overexposed and have less speckle features to be identified (See Fig. \ref{fig:example_images_gain}). This is also evident in their low \gls{mig} values, which are < 37 units and < 1 unit respectively. This lack of features also leads to poor template matching especially in case of gain of 30, where mean error and standard deviation of error is significantly higher as compared to other gain values. The experimental findings suggest that when \gls{mig} ranges between \sim262 to \sim409, the maximum magnitude of errors in position is \SI{5}{\micro\meter}. Here, as well \gls{mig} can be used to quantify contrast in an image, which increases as gain increases from 0 to 4 and then decreases as gain values increase further. But there seems to be no apparent visible difference between images captured for gain values 0, 2 and 4. Additionally, this test does not establish if higher value of \gls{mig} leads to higher accuracy in position measurements.


\section{Analysis on Laser Spot Diameter}
    Changing the laser spot diameter had two clear effects on appearance of \gls{lsp}. With an increase in laser spot diameter, the laser speckle size decreased and speckle density increased, as observed qualitatively from Fig. \ref{fig:laser_spot_dia_example_images}. Although the calculation of these values was beyond the scope of this study, a trend is observed in \gls{mig} values (See Fig. \ref{subfig:laser_spot_dia/mig_laser_spot.png}). Whereby \gls{mig} starts increasing as laser spot diameters increase from \SI{1}{\milli\meter} to \SI{5}{\milli\meter}. This can be attributed to the increasing number of speckles, resulting in more `edges'. But, further tests are necessary to determine the reason of \gls{mig} decreasing with fine speckles and high speckle density, as observed for laser spot diameters of \SI{7}{\milli\meter} and \SI{9}{\milli\meter}.
    
    \vspace{5mm}
    \noindent Another noteworthy observation is that, although \gls{mig} values for this dataset are within the range specified in Section \ref{section:analysis_exposure_time}, there is error in final position measurement for laser spot diameters of \SI{1}{\milli\meter} and \SI{2}{\milli\meter}. This can be attributed to the observed effects of decorrelation when performing template matching, consequently causing drop in confidence value which is larger than other laser spot diameters (See Fig. \ref{subfig:laser_spot_dia/confidence.png}). Notably, this caused errors in position measurements, whereby possible template position detected by \gls{ncc} went amiss as compared to other laser spot diameters (See Fig. \ref{subfig:laser_spot_dia/decorrelation.png}). It is crucial to note that this study does not provide procedure for determining the best size for laser spot diameter; rather, it shows that the diameter has an effect on template matching. Furthermore, this test does not perform accuracy comparison for position measurement between different laser spot diameters.    
    
    
% \vspace{10mm}
% \noindent From comparing Table \ref{table:stats_matrix_gain} and Table \ref{table:stats_matrix_gain_20}, it can be inferred that as \gls{mig} becomes extremely low, errors in position along X-Axis and Y-Axis becomes more apparent. The maximum absolute mean error along X-Axis and along Y-Axis is \SI{5}{\micro\meter} for gain of 0, 2, 4, 6, 8, 10. The positional errors become evident for gain value of 20 and 30, where they are greater than actual displacement of the hexapod.


% \noindent Due to limited literature on influence of laser diameter on position measurement accuracy, it is of necessity to revisit concept of decorrelation. Decorrelation occurs, when the \gls{lsp} undergoes significant changes after a certain amount of translation. This decorrelation also happens during the conducted experiments. Due to the nature of formation of \gls{lsp}, surface roughness plays a pivotal role in determining how laser interacts with the surface and is captured by the camera. This is specially the case, when there is a moving surface, as that has an effect on the underlying reflection. Thus, having an impact on appearance of \gls{lsp}. 
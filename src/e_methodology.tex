\chapter{Methodology}\label{chapter:methodology}    
    \section{Choosing the image quality parameter}
    Due to the homogeneous distribution of speckles in \gls{lsp} image, characterized by minimal variations in gray values among individual speckles \cite{song}, a preference was established for employing a global quality parameter. At present, most quality parameters concern with artificial speckle patterns. Out of them \gls{mig} emerged as a widely accepted parameter for judging speckle pattern quality \cite{hu_ef}. Considering these reasons and its ease of implementation, the decision was made to adopt \gls{mig} as the primary quality parameter for the evaluation of LSP quality.
    
        \subsection*{\glsfirst{mig}}
            \textbf{Intensity Gradient} is the directional change in intensity of a pixel in an image. \gls{mig} can be calculated by using an image gradient operator generally used to perform edge detection. For the use case of this project \emph{Sobel Operator} was used for detecting edges. It was selected because it provides a straightforward calculation of approximated gradient by directly convolving with the underlying image. Because gradients can change in horizontal as well as vertical directions, two sobel operators are needed. Namely, $S_x$ and $S_y$ for X and Y directions respectively.

            \[
                S_x = 
                \begin{bmatrix}
                    -1 & 0 & 1 \\
                    -2 & 0 & 2 \\
                    -1 & 0 & 1 \\
                \end{bmatrix}
                \quad
                S_y = 
                \begin{bmatrix}
                    -1 & -2 & -1 \\
                    0 & 0 & 0 \\
                    1 & 2 & 1 \\
                \end{bmatrix}
            \]

            \vspace{5mm}
            \noindent Sobel operator is a discrete differentiation operator, which calculates approximate gradient of image intensity function. A 3 \times\ 3 Sobel operator was used, because it considers a smaller neighborhood of pixels, is less influenced by outliers, and computations are faster as compared to its 5 \times\ 5 operator counterpart. The gradient calculation in X ($G_x$) and Y ($G_y$) direction is done by convolving $S_x$ and $S_y$ with the original image $A$ (See Eqn. \ref{eqn:g_x} and \ref{eqn:g_y}). Total magnitude of gradient is calculated using the Eqn. \ref{eqn:mag_eqn}. The image quality parameter \gls{mig} is calculated by dividing the magnitude $G$ by size of image (\( m \times n\) pixels) using Eqn. \ref{eqn:mig}. 
            
            %  Then, magnitude of intensity gradient $G_x, G_y$ for each pixel is calculated as given by the Eqn. \ref{eqn:mag_eqn}, after the gradients had been calculated for X and Y directions using Eqn. \ref{eqn:g_x} and Eqn. \ref{eqn:g_y}. The results of the Sobel operation and resulting gradient magnitude after binary thresholding, are displayed in Fig. \ref{fig:sobel.png}. This figure shows how over-/underexposed images have lower \gls{mig} values, because of lesser number of pixels having a higher gradient values (indicated by white pixels). Finally, in order to calculate average intensity gradient for the whole frame, the magnitude of gradients was summed up for all pixels and divided by the total number of pixels inside the frame, as given by Eqn. \ref{eqn:mig}, where $m$ and $n$ denote the number of rows and columns of pixels inside the frame respectively. To calculate average \gls{mig} for a certain pair of exposure time and gain (See Eqn. \ref{eqn:avg_mig}), \gls{mig} across all frames are added and divided by total number of frames. 

            \begin{equation}
                G_x = S_x * A = 
                \begin{bmatrix}
                    -1 & 0 & 1 \\
                    -2 & 0 & 2 \\
                    -1 & 0 & 1 \\
                \end{bmatrix} * A
                \label{eqn:g_x}
            \end{equation}

            \begin{equation}
                G_y = S_y * A = 
                \begin{bmatrix}
                    -1 & -2 & -1 \\
                    0 & 0 & 0 \\
                    1 & 2 & 1 \\
                \end{bmatrix} * A
                \label{eqn:g_y}
            \end{equation}

            \begin{equation}
                G = \sqrt{G_x^2 + G_y^2}
                \label{eqn:mag_eqn}
            \end{equation}

            \begin{equation}
                \text{\gls{mig}} = \frac{1}{m \cdot n} \sum_{i=0}^{m} \sum_{j=0}^{n} G_{ij}
                \label{eqn:mig}
            \end{equation}

            % \begin{equation}
            %     \text{Avg. \gls{mig}} = \frac{\sum_{i=0}^{n} \gls{mig}_i}{\text{Total number of frames}}
            %     \label{eqn:avg_mig}
            % \end{equation}

            % \begin{figure}[h]
            %     \centering
            %     \begin{subfigure}[b]{0.95\textwidth}
            %         \centering
            %         \includegraphics[width=\textwidth]{images/b_fundamentals/sobel_gradient.png}
            %         \caption{Gain: 0, Exposure Time: \SI{150}{\micro\second}}
            %         \label{fig:sobel_gradient.png}
            %     \end{subfigure}
            %     \begin{subfigure}[b]{0.95\textwidth}
            %         \centering
            %         \includegraphics[width=\textwidth]{images/b_fundamentals/sobel_underexposed.png}
            %         \caption{Underexposed \gls{lsp} with Gain: 0, Exposure Time: \SI{20}{\micro\second}}
            %         \label{fig:sobel_underexposed.png}
            %     \end{subfigure}
            %     \begin{subfigure}[b]{0.95\textwidth}
            %         \centering
            %         \includegraphics[width=\textwidth]{images/b_fundamentals/sobel_overexposed.png}
            %         \caption{Overexposed \gls{lsp} with Gain: 0, Exposure Time: \SI{1000}{\micro\second}}
            %         \label{fig:sobel_overexposed.png}
            %     \end{subfigure}
            %     \caption{Grayscale colormaps for three different \glsplural{lsp} after Sobel operations along X and Y directions and resulting magnitude.}
            %     \label{fig:sobel.png}
            % \end{figure}
    
    
    \section{Choosing Correlation Algorithm}        
        As per the literature survey, \gls{ncc} is widely used as a correlation algorithm. The study also utilizes the same algorithm. It should also be made clear, that primary objective of this research is not to compare different template matching or sub-pixel interpolation algorithms, but to investigate, if \gls{mig} is a suitable quality parameter for \gls{lsp} images to assess displacement accuracy. As correlation algorithm is not be changed during the course of the conducted experiments, measurement error analysis should be independent of it. 
            
    
    \section{Displacement Measurement Verification}
        \subsection*{Displacement Calculation}
            \noindent After images are recorded simultaneously for the motion of hexapod to position (\(a_{x\ hexapod}, a_{y\ hexapod}\)), the displacements are calculated in pixel units by \gls{ncc}. Verification of this displacement with hexapod motion is done using the Eqn. \ref{eqn:dist_calc} \cite{charrett_2018}. 

            \begin{equation}\label{eqn:dist_calc}
                \begin{aligned}
                    a_x = \frac{A_x \cdot T_{yy} - A_y \cdot T_{xy}}{T_{xx} \cdot T_{yy} - T_{xy} \cdot T_{yx}} \\
                    a_y = \frac{A_y \cdot T_{xx} - A_x \cdot T_{yx}}{T_{xx} \cdot T_{yy} - T_{xy} \cdot T_{yx}}
                \end{aligned}
            \end{equation}

        \subsection*{Error Calculation}
            The error in measured displacement was calculated by Eqn. \ref{eqn:error_x} and \ref{eqn:error_y}. The mean and standard deviation of error across the 40 repetitions of hexapod motion was calculated by using JASP software \cite{jasp_2023}. The experimental plan in Table \ref{table:exp_plan_exposure_time} and \ref{table:exp_plan_gain} details conditions for exposure time, gain and hexapod motion used for each image acquisition.
            
            % The following equations are used in explanation of results. The distances $a_x$ and $a_y$ are calculated using the Eqn. \ref{eqn:dist_calc}, after the hexapod has been moved to a coordinate $(a_{x\ hexapod}, a_{y\ hexapod})$. The error in position for respective axes is calculated using the following formulae:

            \begin{equation}\label{eqn:error_x}
                \text{Error X (mm)} = a_x - a_{x\ hexapod}
            \end{equation}
            \begin{equation}\label{eqn:error_y}
                \text{Error Y (mm)} = a_y - a_{y\ hexapod}
            \end{equation}
            % \begin{equation}
            %     \text{Error X (\%)} = \frac{\text{Error X (mm)}}{a_{x\ hexapod}} \times 100\%
            % \end{equation}
            % \begin{equation}
            %     \text{Error Y (\%)} = \frac{\text{Error Y (mm)}}{a_{y\ hexapod}} \times 100\%
            % \end{equation}

        % Also, this research does not investigate different sub-pixel interpolation algorithms, as it is not implemented for this research.
    
    
     
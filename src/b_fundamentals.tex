\chapter{Fundamentals}

In this chapter, the principles of important concepts will be explained. This helps, in better understanding of the factors affecting the experimental setup, which will be covered in Chapter \ref{chap:sota}.

    \section*{Laser Speckle Formation}

    Most of the surfaces on a scale of wavelength of light are rough, and when any coherent light (or laser) is incident on those surfaces, it is scattered in many different directions. This is known as \emph{diffuse reflection}. Now, these reflected rays can be captured using a camera. The intensity at any point in the image depends on the interference of the reflected light waves. A constructive and a destructive interference results in a bright and dark spot respectively, thus forming the speckle pattern as shown in Fig. ~\ref{fig:laser_speckle_pattern}.

    \begin{figure}[h]
            \centering
            \includegraphics[width=0.4\textwidth]{images/b_fundamentals/diffuse_reflection.jpg}
            \caption{Diffuse Reflection (\cite{img_diffuse})}
            \label{fig:diffuse_reflection}
    \end{figure}

    \begin{figure}[h]
        \centering
        \includegraphics[width=0.4\textwidth]{images/b_fundamentals/laser_speckle.jpg}
        \caption{A typical laser speckle pattern (\cite{briers})}
        \label{fig:laser_speckle_pattern}
    \end{figure}

    \begin{figure}[h]
        \centering
        \includegraphics[width=0.4\textwidth]{images/b_fundamentals/green_laser.jpg}
        \caption{A laser reflected of a reflective object (here, a part of hard drive) (\textcopyright \ David Bode) (\cite{img_green_laser})}
        \label{fig:green_laser}
        \end{figure}

    \begin{figure}[h]
        \centering
        \includegraphics[width=0.4\textwidth]{images/b_fundamentals/white_paper.jpg}
        \caption{A green laser hitting a white piece of paper (\textcopyright \ David Bode) (\cite{img_white_paper})}
        \label{fig:white_paper}
    \end{figure}

    \vspace{5mm}
    \noindent As a paper is less reflective than a highly polished surface, it undergoes diffuse reflectance as seen from Fig. \ref{fig:white_paper}. On the other hand, for a surface with lower roughness, the reflection would resemble as shown in Fig. \ref{fig:green_laser}.

    \clearpage

    \section*{Diffraction and Airy Disk}\label{section:diffraction}

    In diffraction, light waves bend around the corners of a slit as shown in Fig. \ref{fig:diffraction_slit}. In case of circular aperture inside a camera lens, as a result of diffraction, \emph{airy disk} can form on the image sensor (See Fig. \ref{fig:airy_disk}). The angle $\theta$ at which the first minimum of the light wave, i.e. the first dark ring surrounding the center bright spot, occurs is given by the following formula:

    \vspace{5mm}
    \begin{equation}\label{eqn:objective}
        \sin \theta \approx 1.22 \frac{\lambda}{d}
    \end{equation}

    \vspace{5mm}
    \noindent where,
    \begin{itemize}
        \item $\theta$ is angle at which first minimum occurs
        \item $\lambda$ wavelength of incoming light
        \item \emph{d} is diameter of aperture
    \end{itemize}
    
    \vspace{5mm}
    \noindent Hence, as \emph{d} decreases $\theta$ increases, which means the first dark ring appears farther from the center of the bright spot. Another way to frame this is, the bright spot increases in size.

    \begin{figure}[h]
        \centering
        \includegraphics[width=0.3\textwidth]{images/b_fundamentals/diffraction_slit.png}
        \caption{Numerical approximation of diffraction pattern from a slit of width four wavelengths with an incident plane wave. The main central beam, nulls, and phase reversals are apparent. (\cite{img_diffraction_slit})}
        \label{fig:diffraction_slit}
    \end{figure}

    \begin{figure}[h]
        \centering
        \includegraphics[width=0.3\textwidth]{images/b_fundamentals/airy_disk.jpeg}
        \caption{A real Airy disk created by passing a red laser beam through a \SI{90}{\micro\meter} pinhole aperture with 27 orders of diffraction. (\cite{img_airy_disk})}
        \label{fig:airy_disk}
    \end{figure}


    \section*{\glsfirst{ncc} and Template Matching}\label{section:ncc}

    The overall idea of template matching can be summed with two questions:
    
    \begin{itemize}
        \item If the specific template or pattern present in the image?
        \item If yes, then where is the pattern's location?
    \end{itemize}

    \vspace{5mm}
    \noindent In Fig. \ref{fig:template_matching}, the template on the right, is found in the image on the left inside the red box. The process of template matching is performed by comparing each of the pixel values inside the template to the underlying image, using a technique called \emph{squared euclidean distance}. Here, the difference in pixel values for each pixel location between template and the underlying area in the image is squared and added together. The basic idea is, that if this sum has a low value, that is the probable location of template inside the image. The mathematical expression to perform template matching using \gls{ncc} can be found in greater detail in the paper by \cite{lewis}. In Fig. \ref{fig:wiki_ncc_fig1} and Fig. \ref{fig:wiki_ncc_fig2}, the template (shown as the white and black spotted pattern) is being compared with the image that contains this pattern inside a black rectangle. In \gls{ncc}, the match between template and image is shown by a peak in cross-correlation value.

    \begin{figure}[h]
        \centering
        \includegraphics[width=0.6\textwidth]{images/b_fundamentals/template_matching.jpeg}
        \caption{Template/Patch Image (\cite{ralph_caubalejo})}
        \label{fig:template_matching}
    \end{figure}
    
    \begin{figure}[h]
        \centering
        \includegraphics[width=0.7\textwidth]{images/b_fundamentals/wiki_ncc_fig1.png}
        \caption{Normalized cross-correlation - sliding a template over an image. Taken from (\cite{wiki_ncc})}
        \label{fig:wiki_ncc_fig1}
    \end{figure}

    \begin{figure}[h]
        \centering
        \includegraphics[width=0.7\textwidth]{images/b_fundamentals/wiki_ncc_fig2.png}
        \caption{Normalized cross-correlation - sliding a template over an image. Taken from (\cite{wiki_ncc})}
        \label{fig:wiki_ncc_fig2}
    \end{figure}